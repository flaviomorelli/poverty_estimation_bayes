\chapter{Theoretical Background}
This chapter is an overview of the key theoretical concepts used in the rest of this paper.
The main aim is to provide a short review to the reader, but not to discuss every theoretical aspect in detail.
Therefore, additional literature references are mentioned throughout the chapter.
Note that more specific theoretical ideas will be introduced in the relevant step of the workflow in chapter \ref{ch:workflow}.
%Because this paper develops a model iteratively, this structure should make it easier to understand the incremental changes to the model, instead of presenting all theoretical concepts at once in one chapter.

%The first section provides a short summary of Bayesian statistics, starting with Bayes' theorem and discussing Bayesian model evaluation.
%The second half of this chapter presents Bayesian unit-models used for small area estimation.
%It defines poverty indicators and shows how to estimate them in a Bayesian context.



\section{Overview of Bayesian statistics}

This section provides a short introduction to the main concepts of Bayesian statistics.
An exhaustive exposition of Bayesian statistics can be found in sources such as \cite{gelman_bayesian_2014} or \cite{mcelreath_statistical_2020}.
The first part presents the notation of Bayes' theorem and introduces the components of a Bayesian model, while the second part summarizes different ways of evaluating Bayesian models that will be used throughout this paper to compare multiple models.
The more technical topic of Bayesian computation is introduced in appendix \ref{ch:computation} – specifically, the Markov Chain Monte Carlo (MCMC) and Hamiltonian Monte Carlo (HMC) algorithms.
In general, the reader should assume that there were no major computational problems in the models presented, unless otherwise specified.


\section{Bayesian Inference and Terminology}

Inference is the process of analyzing population parameters based on samples from the population.
Because the sample is an imperfect representation of the population, there is always uncertainty associated with parameter inference.
In frequentist statistics, the common approach is to calculate point estimates of the true parameters from the data.
Uncertainty about the estimate comes from the randomness of the data in the sample and it is usually quantified by confidence intervals, which are based on distributional assumptions such as normality.
Bayesian inference takes a different approach: the true parameter is not a constant, but a random variable with a distribution, while the data is assumed to be fixed.
The aim is to estimate the distribution of the population parameter $\theta$ given the data $y$, i.e. $p(\theta|y)$.
The derivation of the distribution is based on Bayes' theorem
\begin{gather*}
	\displaystyle p(\theta | y) \overset {\text{(a)}}{=}  \frac {p(\theta, y)}{p(y)} \overset {\text{(b)}}{=} \frac {p(\theta) p(y|\theta)}{\int p(\theta) p(y|\theta)d\theta}.
\end{gather*}
Here, $\theta$ is a vector of parameters and vector $y$ represents the data, which can be anything from a single variable to a regression design matrix with its respective outcome variable.
Inference is based on the distribution $p(\theta | y )$ – the \textbf{posterior distribution}.
The first equality (a) is the definition of conditional probability.
By using the definition of conditional probability again, it is possible to decompose the joint distribution $p(\theta, y)$ into $p(\theta)$ and $p(y|\theta)$ in the second inequality (b).
$p(\theta)$ is the \textbf{prior distribution} which encodes knowledge about the parameters prior to collecting the data. One example of prior information in the context of a socio-economic survey would be knowing the past average per capita income or the Gini coefficient of a country from sources such as a textbook or existing research papers before even collecting current data.
Prior information also includes well-known facts about distribution parameters – e.g. the variance cannot be negative.
This information can be used to parametrize the prior distribution.
The \textbf{likelihood} is given by $p(y | \theta)$ and it reflects the modelling of the data generating process.
A simple example is a binomial distribution that models the number of sucesses $y$ in $n$ tosses of a coin, given the probability $\theta$.
Both the prior and the likelihood are the main ingredients of Bayesian inference.

The term in the denominator $p(y)$ is the \textbf{marginal likelihood} which gets its name from marginalizing $\theta$ out of the expression in the numerator by integration: $\int p(\theta) p(y|\theta)d\theta$.
The marginal likelihood is a normalizing constant that ensures that $p(\theta|y)$ is a proper distribution, i.e. integrates to 1.
The integral is intractable even for very simple models and analytic solutions exist mostly for some cases of conjugate priors.
Therefore, it is necessary to use Monte Carlo methods or variational inference to estimate the posterior (see chapter 2.3). The popularity of Bayesian methods has increased with the availabilty of computational power that makes such estimation methods more accessible.
Nevertheless, the marginal likelihood is more than a simple normalizing constant.
By sampling from $p(y)$, the marginal likelihood can be used to generate data from the model, even before observing any data.
Therefore, it is also called the \textbf{prior predictive distribution}.
To generate new data  $\tilde y$ from the posterior model, replace the prior $p(\theta)$ with the posterior $p(\theta|y)$ in the definition of $p(y)$ and the result is $p(\tilde y|y) = \int p(y | \theta) p(\theta|y) d\theta$, where $p(\tilde y | y)$ is the \textbf{posterior predictive distribution}.
Data generated from the prior predictive and posterior predictive distributions can be used to check model quality.
This will be further discussed in chapter 2.4.
A more detailed description of the concepts presented in this section can be found in chapters 1 and 2 of \cite{gelman_bayesian_2014}.



\section{Evaluation of Bayesian models}

There are numerous ways to evaluate Bayesian models.
(PIIRONEN VEHTARI 2017) provides an overview of evaluation methods that quantify predictive power of a model, which can be used to compare different models.
On the other hand, Bayesian inference provides two additional checking tools: prior and posterior predictive checks.
The main idea here is to generate numerous samples either from the prior predictive or posterior predictive distribution described in section XY.
Thus, it is possible to check how the model behaves before and after fitting the data and whether the generated samples are in a plausible range of the dependent variable.
This section focuses on PSIS-LOO as a measure of predictive power and includes a brief discussion of prior and posterior predictive checks.

Given some data $y$ and future observations $\tilde y_i, i = 1, ..., N$, where $N$ is the original sample size, the quality of the predictive distribution can be defined in terms of a utility function in terms of the logarithmic score (Good, 1952, Piironen/Vehtari 2017)
\begin{gather*}
    u(\tilde y_i) = \displaystyle \sum_{i = 1}^N\log p(\tilde y_i|y).
\end{gather*}
However, as $\tilde y$ is unobserved, it is necessary to marginalize it out of $u$, thus getting the expected utility
\begin{gather*}
    \text{elpd} =
    \displaystyle \sum_{i = 1}^N E[\log p(\tilde y_i| y)] =
    \displaystyle \sum_{i = 1}^N \int p_t(\tilde y_i) \log p(\tilde y_i| y) d \tilde y_i,
\end{gather*}
where $p_t$ is the true data generating distribution and elpd stands for expected log pointwise predictive density for a new data set.
As $p_t$ is unknown, it is not possible to calculate the elpd directly.
An unbiased estimate for the elpd is given by
\begin{gather}
    \text{elpd}_{\text{loo}} =
    \displaystyle \sum_{i = 1}^N \log p(\tilde y_i| y_{-i}),
    ~~~ p(\tilde y_i| y_{-i}) = \displaystyle \int p(y_i | \theta)p(\theta|y_{-i})d\theta.
\end{gather}
Here, $p(\tilde y_i| y_{-i})$ is the leave-one-out predictive distribution given $y_{-i}$, i.e. the data without the $i$-th observation.
In practice, $\text{elpd}_{\text{loo}}$ is estimated efficiently by Pareto-smoothed importance sampling without having to refit the model $N$ times and is therefore referred to as PSIS-LOO.
A higher PSIS-LOO indicates a model with a higher predictive power.
Pareto smoothing is not only useful for the efficient estimation of elpd$_{\text{loo}}$, but it also provides a diagnostic to whether the estimated PSIS-LOO can be trusted.
Specifically if the shape parameter $k$ of the Pareto distribution is higher than 0.7, then PSIS-LOO is not reliable(Vehtari/Gelman 2017).
This might be alleviate with the moment match method (Bürkner, Vehtari), but it can also indicate that the model does not deal with outliers well.
Note that a utility function that quantifies predictive power takes into account the distributional characteristics of the model.
In contrast, loss function approaches such as the root mean squared error (RMSE) or the mean absolute error (MAE) measure the distance between the predicted values and the true values, but are distribution-agnostic.

Beyond PSIS-LOO, prior and posterior predictive checks are useful to asses model quality.
The main idea is to generate multiple samples from the prior predictive distribution $\int p(\theta) p(y|\theta)d\theta$ or from the posterior predictive distribution $\int p(y | \theta) p(\theta|y) d\theta$.
With these samples, it is possible to check how similar the simulated distribution is to the original distribution and also whether the simulations are within a reasonable range.
This similarity check can be done for example with histograms or KDE plots when using the full distribution.
As Bayesian models produce a full distribution, it is also possible to check how certain summary statistics (median, IQR, variance, quantiles etc.) vary between the simulated samples and compare them to the summary statistics in the original distributions. (GELMAN et al 2014)
However, these statistics should be ancillary in the sense that they should test something different than parameter fit.
While a linear model fits the mean and the variance quite well, this is not the case for a Poisson regression.
Therefore, checking the mean and the variance in the Poisson case will reveal potential problems in the model (Stan User guide 27.3).
A practical application of prior and posterior check will be shown in section X and section Y.





\section{Poverty indicators and small area estimation}

After the short introduction to the basics of Bayesian inference, this section describes how hierarchical Bayesian models are used in small area estimation – with a special focus on unit-level models.
It then defines the poverty indicators that are at the center of this paper and presents an algorithm to estimate those indicators with hierarchical Bayesian models.

\subsection{Bayesian unit-level model for small area estimation}

\textit{Small area estimation} is a field of survey statistics that deals with prediction in areas for which there is little or no information.
The terms \textit{area} or \textit{domain}, – usually used as synonyms –, do not necessarily imply a geographic area.
More generally, it can denote any subgroup of a population arising from disaggregation – by gender, region, ethnicity, etc.
A typical scenario for small area estimation arises in surveys, where the representative sample is small compared to the whole population.
If indicators are needed at a finer disaggregated level (e.g., by municipality), the sample size in each area can become extremely small (under 20 observations) and there might be areas with no observations at all.
To improve predictions for these small areas, the models borrow strength from additional data sources such as a census or a register.
Depending on the data available, there are two types of small area models.
Area-level models such as the Fay-Herriot use aggregated data for each area to improve direct estimators.
On the other hand, unit-level models need information at the individual or household level to generate predictions \citep[Chapter 1 and 2]{rao_small_2015}.

\cite{molina_small_2014} formulate a Bayesian unit-level model, which is referred to as the Hierarchical Bayes (HB) model and is based on the Battese-Harter-Fuller (BHF) model \citep{battese_error_1988}:
\begin{equation}
    \label{eq:hb_rao}
	\begin{split}
	y_{di} |\boldsymbol \beta, u_d, \sigma_e & \sim \mathcal N(\boldsymbol{x'}_{di} \boldsymbol{\beta}+ u_d, \sigma_e), ~ d = 1, ..., D, ~ i = 1, ..., N_d \\
	u_d | \sigma_u & \sim \mathcal N(0, \sigma_u), d = 1, ..., D \\
	p(\boldsymbol \beta, \sigma_u, \sigma_e) & = p(\boldsymbol \beta) p(\sigma_u)p(\sigma_e) \propto p(\sigma_u)p(\sigma_e).
	\end{split}
\end{equation}
The first distribution defines the likelihood and the last two lines define the prior, taking into account conditional dependencies.
$D$ is the number of domains and $N_d$ is the number of observations for domain $d = 1, ..., D$, whereas $y_{di}$ and $\boldsymbol{x}_{di}$ are respectively the dependent and independent variables for area $d$ and observation $i$ in that area.
$\boldsymbol \beta$ is a vector of regressor coefficients common to all areas.
The effect for area $d$ is given by $u_d$ and the common variance parameter for all area effects is $\sigma_u$.
The variance parameter at the individual level is given by $\sigma_e$.
Note that $\boldsymbol \beta, \sigma_u$ and $\sigma_e$ are assumed to be independent.
Their priors are $p(\boldsymbol \beta), p(\sigma_u)$ and $p(\sigma_e)$ respectively.

However, model \ref{eq:hb_rao} does not take full advantage of Bayesian modelling.
By taking the normal distribution as the likelihood, the model faces the same limitations of a frequentist linear regression and will not be able to deal with heavy-tailed data.\footnote{\cite{morelli_hierarchical_2021} dealt with heavy tails by using a Student's $t$-distribution as the likelihood.}
Moreover, the prior distributions in \cite{molina_small_2014} are non-informative (flat), i.e., they are proportional to a constant and not a proper distribution.
This poses two problems.
First, a Bayesian model with flat priors is not a generative model in the sense that it is not possible to simulate new observations from the prior predictive distribution.
Second, it does not take advantage of the extra control that priors provide over the model when modelling relation between parameters.
Therefore, \cite{morelli_hierarchical_2021} reformulates the model \ref{eq:hb_rao} as follows with a Student's $t$-likelihood:
\begin{equation}
	\begin{split}
		y_{di} |\boldsymbol \beta, u_d, \sigma_e & \sim
            \text{Student}(\boldsymbol{x'}_{di} \boldsymbol \beta + u_d,\ \sigma_e\ , \nu),\ d = 1, ..., D,\ i = 1, ..., N_d, \\
		u_d | \sigma_u & \sim \mathcal N(0, \sigma_u),\ d = 1, ..., D, \\
		\beta_k & \sim \mathcal N(\mu_k, \sigma_k),\ k = 1, ..., K,\\
		\sigma_u & \sim Ga(2, a), \\
		\sigma_e & \sim Ga(2, b), \\
		\nu & \sim Ga(2, 0.1). \\
	\end{split}
	\label{eq:mod_hb}
\end{equation}
The $t$-distribution has an extra parameter $\nu$, the degrees of freedom, that has an impact on its excess kurtosis and consequently also on the variance.
Note that in this case there is only one $\nu$ for all areas.
Because the excess kurtosis converges to zero as $\nu \rightarrow \infty$  (for $\nu \approx 50$ the excess kurtosis is just 0.1), a gamma distribution with shape 2 and $0.1$ as the rate parameter is used.
This forces $\nu$ to be positive and places more weight on the areas of $\nu$ for which the $t$-distribution is leptokurtic, while still allowing values where the likelihood is close to normal (the 95\% quantile is just below 50, and for $\nu \ge 50$ there is little difference between a Gaussian and a Student's $t$-distribution).
Note that $\sigma_e$ is a scale parameter of the Student's $t$-distribution and is not equal to its standard deviation.
The standard deviation of the likelihood is given by the relation $\sigma = \sigma_e \cdot \frac{\nu}{\nu - 2}$, which makes clear that the variance is only finite for $\nu > 2$.
The gamma distribution is chosen for the scale parameters $\sigma_u$ and $\sigma_e$, as the standard deviation cannot be negative.
The shape parameter is set to 2 in line with \cite{gelman_prior_2020}, which makes the distribution clearly skewed to the right.
$a$ and $b$ are positive constants that define the rate parameters of the gamma distributions.
This have to be chosen according to the scale of the dependent variable $y_{di}$ for each specific data set.
$K$ is the number of coefficients in the regression, and $k = 1$ is the intercept.
Thus, according to \ref{eq:mod_hb} the prior can in theory be set independently for each coefficient in $\boldsymbol \beta$.
The exact prior parameters for the coefficients will be discussed in the next sections based on the variables from the data set.


\cite{molina_small_2014} also present a reparametrized version of model \ref{eq:hb_rao} with $\rho = \sigma_u(\sigma_u + \sigma_e)^{-1} $ to avoid using MCMC methods\footnote{An in-depth explanation of MCMC can be found in appendix \ref{ch:computation}.}.
Avoiding MCMC should not be a major concern due to the many developments in the field of Bayesian computation since 2014.
While the reparametrized model may simplify estimation under certain circumstances, it comes at the cost of model flexibility. It is not straightforward to imagine how their reparametrized version could be estimated without MCMC after changing the likelihood or prior distributions.
Further information on Bayesian computation can be found in appendix \ref{ch:computation}.
The next section defines poverty indicators based on income and describes an algorithm to calculate them based on Bayesian models such as \ref{eq:mod_hb}.
%\section{Income Data and Poverty Indicators}
Income data tend to be challenging to model, because they usually follow a skewed, leptokurtic distribution: most people earn an average or lower-than-average income, while only a few have very high incomes. This skewed distribution presents challenges from a statistical point of view. Many frequentist regression models are based on the assumption of normality, which is clearly violated when the data is skewed. 
A common fix is to transform the data with the natural logarithm which is easy to backtransform to the original scale with the exponential function. While this tends to make the distribution more symmetric, there might be some skewness left. Another problem is that the kurtosis of the log-transformed income rarely matches the kurtosis of a normal distribution. 

\cite{rojas_perilla_data_2020} propose different data-driven transformations beyond the basic log-transformations which bring the outcome variable closer to a normal distribution. While data-driven transformations are effective, the aim of such methods is to better fit the normality assumption of frequentist linear models. This assumption limits the freedom to choose alternative modelling strategies for the data. In the Bayesian context, there are no predetermined assumptions concerning the models. On the contrary, the researcher can include prior knowledge though the choice and parametrization of the priors and has some flexibility regarding the likelihood of the model. A normal likelihood can be easily replaced by a Student's $t$-distribution or a Cauchy distribution if needed. Moreover, even if the likelihood follows a normal distribution, the prior distributions can make the posterior non-normal. As already discussed, another key advantage of Bayesian models is the ability to capture complex multi-level structures by using the flexible conditional probability structure of the priors.

There are two approaches on how to model unimodal, skewed, leptokurtic data such as income. One approach to model skewed data is to use a skewed distribution such as a Pareto, a skew normal or a Gamma distribution for the likelihood. However, these can cause problems for MCMC for two reasons. First, MCMC can have trouble with the fat tail to the right of the distribution mode. Second, it might not perform as well with data that is bounded from below by zero. Another approach is to do a log or log-shift transformation of the outcome variable and model it with a symmetric distribution such as a Normal, Student-t, Logistic or Cauchy. 
The log-shift transformation has the advantage of being mathematically simpler than power transformations and it is already widely used for income data in fields such as econometrics. While there might be some skewness left after taking the logarithm, this can still be minimized by choosing an appropriate shift term. Moreover, the shift term eliminates values that are close to zero which can be a problem in the logarithmic scale, since $\underset{x \rightarrow 0}\lim \log(x) = -\infty$. With almost no skewness left after the log-shift transformation, the main challenge is to model the excess kurtosis of the variable. Therefore, it is reasonable to use a distribution that allows for fat tails. This paper focuses on the second approach and explores how to best model the excess kurtosis of income in the log-scale.

Using income data, it is possible to examine how this approach applies to poverty indicators such as head count ratio (HCR) and poverty gap (PGAP). HCR and PGAP are based on the Foster-Greer-Thorbecke indicator \citep{foster_class_1984}:
\begin{gather*}
   F_d(\alpha, t) = \displaystyle \frac 1 {N_d} \sum_{i=1}^{N_d}\left( \frac{t - y_{di}}{t} \right)^\alpha I (y_{di} \le t), 
   \hspace{1cm}\alpha = 0, 1, 2,
\end{gather*}
where $t$ is the poverty line (60\% of median income), $y_{di}$ is the income for the $i$-th household in area $d$ and $I(\cdot)$ is the indicator function. If $\alpha = 0$, then $F_d$ is the HCR – in other words the proportion of households below the poverty line in area $d$. $F_d$ quantifies poverty intensity (PGAP) when $\alpha = 1$, i.e. it measures by how much poor people are below the poverty line on average. $\alpha = 2$ defines poverty severity, which will not be considered in this paper. After discussing the problems with income data and defining the poverty indicators of interest, the next step is to define how income data can be modelled. 





\subsection{Estimating poverty indicators with Bayesian models}

Based on model \ref{eq:mod_hb}, it is possible to generate synthetic income data. The HB estimator for the poverty indicator $F_d$ described in section 3.1 is given by the following algorithm, where $S$ denotes the total number of MCMC samples: \\
\code{For $s = 1, ..., S$}
\begin{enumerate}
    \itemsep -6mm
    \item Sample $\boldsymbol {\hat \beta^{(s)}}, \hat u_d^{(s)}, \hat \sigma_e^{(s)}, \hat \sigma_u^{(s)}[, \hat \nu^{(s)}]$ from the posterior distribution.\\
    \item Sample $\tilde y_d^{(s)}|y$ from the posterior predictive distribution. There are two cases:
    \begin{enumerate}
        \itemsep -8mm
        \item If municipality $d$ is in-sample, then sample $\tilde y^{(s)}_d|y$ from  $\mathcal L (\boldsymbol{x'}_{di} \boldsymbol {\hat \beta^{(s)}} + \hat u_d^{(s)}, \hat \sigma_e^{(s)}[, \hat \nu^{(s)}])$. \\
        \item If municipality $d$ is out-of-sample, first sample $\tilde u_d^{(s)}$ from $\mathcal N(0, \hat \sigma_u^{(s)})$ and then sample $\tilde y^{(s)}_d|y$ from $\mathcal L (\boldsymbol{x'}_{di} \boldsymbol {\hat \beta^{(s)}} + \tilde u_d^{(s)}, \hat \sigma_e^{(s)}[, \hat \nu^{(s)}])$.\\
    \end{enumerate}
    \item Calculate the poverty line $t^{(s)} = 0.6 \cdot median(\tilde y ^{(s)})$. Then calculate $F_d^{(s)}(\alpha, t^{(s)})$ for each $d$ based on $\tilde y_{d}^{(s)}|y$.\\
    \item Finally, $\hat F_d^{HB} = \displaystyle \frac 1 S \sum_{s = 1}^S F_d^{(s)}$ and $\hat \sigma^{HB}_d = \displaystyle\sqrt{ \frac{1}{S-1}  \sum_{s = 1}^S \left( F_d^{(s)} - \hat F_d^{HB} \right)^2}$.

\end{enumerate}
Note that the poverty line $t^{(s)}$ is based on $\tilde y^{(s)} = (y^{(s)}_1|y, ..., y^{(s)}_d|y)$ – the generated income for $all$ areas –, not just for the income of area $d$ given by $\tilde y_d^{(s)}$. Therefore, there is only one poverty line for each MCMC sample $s$. This is a similar procedure to \cite{rojas_perilla_data_2020}, adapted to the Bayesian context. Having discussed the HB model and the likelihoods that can be chosen, it is necessary to compare the different approaches and determine the best one to model income, before being able to estimate poverty indicators.

