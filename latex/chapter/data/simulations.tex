\section{Simulation scenarios}
\label{ch:simulations}
Working with simulations is one key step of iterative model improvement in a Bayesian workflow \citep{gelman_bayesian_2020}.
Testing models against synthetic data lets the researcher check and understand potential problems with the methods.
While a model that works well with simulated data is not guaranteed to work well with real-world data,
a model that does not work with simulated data is certain to fail in a real application.
This will become clearer in the next chapter on Bayesian workflow.

As described in section \ref{ch:difficulties}, income data is characterized by a unimodal, right-skewed and leptokurtic distribution.
To mimic these characteristics, three simulation scenarios based on \cite{rojas_perilla_data_2020} are proposed.
The first one – the \textit{log-scale} scenario – is defined so that the logarithm of simulated income is roughly normal:
\begin{equation}
    \begin{split}
        u_d & \sim \mathcal N(0, 0.4), ~~ d = 1,...,D,\\
        \varepsilon_{di} & \sim \mathcal{N}(0, 0.3), i = 1,...,N,\\
        \mu_{dk} & \sim \mathcal{U}(2, 3), k = 1,...,K,\\
        \Sigma_{mn} & = \begin{cases} 1, ~~ m = n,~~m = 1,...,K, n = 1,...,K, \\ \rho,  ~~ \text{otherwise},  \end{cases}
            \\
        \boldsymbol x_{di}  &\sim \mathcal N (\boldsymbol \mu_{d}, \Sigma) ,
            ~~ \boldsymbol \mu_{d} = (\mu_{d1}, ..., \mu_{dK}),\\
        y_{di} & = \exp(5 + 0.1 \cdot \boldsymbol x_{di}  + u_d + \varepsilon_{di}),\\
    \end{split}
    \label{eq:log_scenario}
\end{equation}
where $K$ is the number of regressors, $D$ is the number of domains, and $N = N_1 + ... + N_D$ is the total number of observations.
$\Sigma$ is the covariance matrix and it assumes that all covariates have unit variance. $\rho$ controls the correlation between independent variables and is set to 0.2.
A certain varation among areas is achieved through the vector $\boldsymbol \mu_d$, which contains the means for all covariates in area $d$.
Thus, some covariates are consistently higher or lower in a given area, creating a clear profile for each area.
Because the focus is on prediction, the regression coefficients are fixed and equal for all covariates (e.g., 0.1 in the log-scale scenario), but this is only done for simplicity.
There are two additional scenarios – the \textit{Pareto} and the \textit{GB2}.
The GB2 scenario can be formulated as
\begin{equation}
    \begin{split}
        u_d & \sim \mathcal N(0, 500), ~~ d = 1,...,D,\\
        \varepsilon_{di} & \sim \mathcal{GB}2(2.5, 18, 1.46, 1700), i = 1,...,N,\\
        \mu_{dk} & \sim \mathcal{U}(-1, 1), k = 1,...,K,\\
        \boldsymbol x_{di}  &\sim  \mathcal N  (\boldsymbol \mu_{d}, \Sigma) ,
        ~~ \boldsymbol \mu_{d} = (\mu_{d1}, ..., \mu_{dK}),\\
        \tilde{\varepsilon}_{di} & = \varepsilon_{di} - \bar \varepsilon,  \\
        y_{di} & = 9000 - 250 \cdot \boldsymbol x_{di} + u_d + \tilde \varepsilon_{di}, \\
    \end{split}
    \label{eq:gb2_scenario}
\end{equation}
where $\mathcal{GB}2$ is the generalized beta distribution of the second kind with four parameters usually referred to as $a, b, p, q$. $\Sigma$ is defined as in scenario \ref{eq:log_scenario} with $\rho = 0.2$. Note that $\varepsilon_{di}$ needs to be centered, as they have a non-zero mean. The third and last scenario has a Pareto error term:
\begin{equation}
    \begin{split}
        u_d & \sim \mathcal N(0, 500), ~~ d = 1,...,D,\\
        \varepsilon_{di} & \sim \text{Pareto}(3, 2000), i = 1,...,N,\\
        \mu_{dk} & \sim \mathcal{U}(-3, 3), k = 1,...,K,\\
        \boldsymbol x_{di}  &\sim  \mathcal N (\boldsymbol \mu_{d}, \Sigma) ,
        ~~ \boldsymbol \mu_{d} = (\mu_{d1}, ..., \mu_{dK}),\\
        \tilde{\varepsilon}_{di} & = \varepsilon_{di} - \bar \varepsilon,  \\
        y_{di} & = 12000 - 350 \cdot \boldsymbol x_{di}    + u_d + \tilde \varepsilon_{di}.\\
    \end{split}
    \label{eq:pareto_scenario}
\end{equation}
Here again, $\tilde \varepsilon_{di}$ are the centered unit-level residuals.
Note that all fixed parameters in the three scenarios (e.g., the standard deviation of the area-level random effect $u_d$) are chosen so that the simulations are in a realistic range in line with the income variable \code{icptc} from the Mexican survey.
This means roughly that no simulation should be above the tens of thousands.

The three scenarios provide a variety of characteristics against which to test the models in the workflow.
First, the log-scale scenario is multiplicative, while the GB2 and Pareto scenarios are additive.
This is useful to check whether a model with a multiplicative likelihood also can approximate a model whose generating process is additive, and vice versa.
Moreover, the logarithmic transformation of $y_{di}$ should be roughly normal, while the GB2 and Pareto scenarios should display a higher excess kurtosis after a logarithmic transform, which is often observed in real-world income data.

Nevertheless, there are two main limitations with the simulation scenarios.
Firstly, the area-level intercepts $u_d$ are assumed to be independent, by defining the standard deviation as a constant.
This is unrealistic, as there are similarities and differences between areas that are likely to manifest themselves as correlations.
A clear example would be two neighboring geographic areas with highly intertwined economies.
Secondly, some covariates might have different effect sizes depending on the area.
For example, a higher level of education could have a much stronger effect on income in an urban region, where there is a higher demand for specialized, well-paying jobs.
In contrast, most jobs in a rural region are less likely to require high skills and therefore offer lower pay.
Here, income not only depends on the quality of labor supply, but also on the job market demands in a given area.
For the present paper, these two limitations are acceptable.
On the one hand, there are many ways of including spatial correlation in the simulation scenarios (see section \ref{ch:area_corr}) and it is not clear whether a decision for one type of correlation might turn out to be unlike the spatial dependencies present in real-world data.
On the other hand, random slopes are not considered in the Bayesian workflow of the next chapter.
Extensions to the simulation scenarios such as spatial correlation and random slopes might play an important role in developing a model for poverty estimation, but this question is left for further research.