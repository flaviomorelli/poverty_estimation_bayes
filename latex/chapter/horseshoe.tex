\section{Variable selection with the horseshoe prior}

A key step in specifying the model is deciding which variables to use.
However, this is at the same time one of the more challenging steps.
A simple such as forward or backward regression might be problematic (why? reference).
The ideal solution would be to estimate models with all possible variable combinations and then compare the predictive quality of the models – e.g., through cross-validaton.
However, this solution is not feasible computationally due to the very large number of possible models.
Another approach is to use shrinkage methods to check which variables can be left out without significantly compromising the predictive power of the model.
If a coefficient is very close to zero, it is a sign that the respective variable might be removed.
In this section, the regularized horseshoe prior \citep{piironen_sparsity_2017} is presented and then used to select relevant variables. The selection is done by comparing the PSIS-LOO of models with the less impactful (CHANGE WORD) variables removed.

\subsection{Horseshoe prior: theory}

In the Bayesian framework, shrinkage can be best understood in the context of the prior.
The main idea is to have a narrow region of high density around zero that shrink coefficients,
while at the same time including fat tails to allow some coefficients to deviate from zero over a wider range.
In its most extreme formulation, it corresponds to the spike-and-slab prior introduced originally by (CITATION).
While this prior has a hight theoretical relevance, it leads to computational problems (WHICH?).
The horseshoe prior, originally introduced by \cite{carvalho_horseshoe_2010}, aims to solve these problems.
The horseshoe prior can be formulated as follows:

In practice, the original horseshoe prior leads to computational problems in the form of HMC divergences.
 \cite{piironen_sparsity_2017} address this problem by introducing a regularized horseshoe prior.

