\chapter{Discussion}

\section{Adequacy of simulation scenarios}

The use of simulated data is a central part of Bayesian workflow according to \cite{gelman_bayesian_2020}.
However, it is also a time-consuming step.
In general, it is hard to create a scenario or multiple scenarios that capture the main features of the data, while at the same time not defining overly complicated models or generating so many scenarios that it is a burden for the researcher to work with them.

The models used in this paper are an extension of the models in \cite{rojas_perilla_data_2020}, mainly through the addition of correlated covariates.
Nevertheless, there are some critical aspects that can be reconsidered.
First, there might be room for improving the assumed distributions in the scenarios.
On th one hand, the GB2 and Pareto scenarios add a skewed error term to covariates and random effects that are normal.
The result is a relatively symmetric distribution with a long right tail.
To make both scenarios more realistic, the distributions of the regressors could be more varied, so as to avoid a strong symmetry before adding the error term.
On the other hand, the log-scale scenario assumes normality in the logarithmic scale.
While this is a good sanity check for the results, it is unlikely to reflect the difficulty of fitting real-world data.
A simple modification to avoid this strong assumption, would be again to change the normal distribution of the regressors to a heavier tailed Student's $t$ whith different the degrees of freedom for each covariate.
Second, a majority of variables found in a survey or a census are categorical.
Neither \cite{rojas_perilla_data_2020} nor the present paper include categorical variables in the simulations.
Therefore, there are probably some blind spots in the analysis of this paper, regarding difficulties that might arise from the use of categorical covariates.
Third, the area-level effects are assumed to be independent – an unrealistic assumption.
Many different ways of including correlation were already mentioned in Chapter X: an LKJ prior, an autoregressive prior (SAR, IAR, ICAR, etc.) or using a random walk between areas.
While choosing any one of those procedures can place a strong assumption on the simulations, it should at least help the researcher to show, whether the model can capture correlations at the area level.
Finally, this paper ignored the issue of problematic regressors.
The correlation between covariates is moderate, only 0.2.
Fitting a model that does not overfit is more challenging with a higher correlation between regressors and also with covariates that are highly correlated with the dependent variable.
Including this suggestions into the simulations scenarios, can provide a more realistic setting to test the models.

At the same time, it is crucial not to overcomplicate the models and not to have too many scenarios.
In this paper, the GB2 and the Pareto scenarios are both additive and generate dependent variables that are similar in shape.
To simplify model analysis, it might be easier to limit the scenarios to one multiplicative scenario and one additive scenario, either with a GB2 or a Pareto error.

LOGSCALE is MOST REALISTIC! WHAT TYPE OF SKEWNESS DO WE HAVE?