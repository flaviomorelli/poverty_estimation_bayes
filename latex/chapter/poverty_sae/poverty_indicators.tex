\section{Income Data and Poverty Indicators}
Income data tend to be challenging to model, because they usually follow a skewed, leptokurtic distribution: most people earn an average or lower-than-average income, while only a few have very high incomes. This skewed distribution presents challenges from a statistical point of view. Many frequentist regression models are based on the assumption of normality, which is clearly violated when the data is skewed. 
A common fix is to transform the data with the natural logarithm which is easy to backtransform to the original scale with the exponential function. While this tends to make the distribution more symmetric, there might be some skewness left. Another problem is that the kurtosis of the log-transformed income rarely matches the kurtosis of a normal distribution. 

\cite{rojas_perilla_data_2020} propose different data-driven transformations beyond the basic log-transformations which bring the outcome variable closer to a normal distribution. While data-driven transformations are effective, the aim of such methods is to better fit the normality assumption of frequentist linear models. This assumption limits the freedom to choose alternative modelling strategies for the data. In the Bayesian context, there are no predetermined assumptions concerning the models. On the contrary, the researcher can include prior knowledge though the choice and parametrization of the priors and has some flexibility regarding the likelihood of the model. A normal likelihood can be easily replaced by a Student's $t$-distribution or a Cauchy distribution if needed. Moreover, even if the likelihood follows a normal distribution, the prior distributions can make the posterior non-normal. As already discussed, another key advantage of Bayesian models is the ability to capture complex multi-level structures by using the flexible conditional probability structure of the priors.

There are two approaches on how to model unimodal, skewed, leptokurtic data such as income. One approach to model skewed data is to use a skewed distribution such as a Pareto, a skew normal or a Gamma distribution for the likelihood. However, these can cause problems for MCMC for two reasons. First, MCMC can have trouble with the fat tail to the right of the distribution mode. Second, it might not perform as well with data that is bounded from below by zero. Another approach is to do a log or log-shift transformation of the outcome variable and model it with a symmetric distribution such as a Normal, Student-t, Logistic or Cauchy. 
The log-shift transformation has the advantage of being mathematically simpler than power transformations and it is already widely used for income data in fields such as econometrics. While there might be some skewness left after taking the logarithm, this can still be minimized by choosing an appropriate shift term. Moreover, the shift term eliminates values that are close to zero which can be a problem in the logarithmic scale, since $\underset{x \rightarrow 0}\lim \log(x) = -\infty$. With almost no skewness left after the log-shift transformation, the main challenge is to model the excess kurtosis of the variable. Therefore, it is reasonable to use a distribution that allows for fat tails. This paper focuses on the second approach and explores how to best model the excess kurtosis of income in the log-scale.

Using income data, it is possible to examine how this approach applies to poverty indicators such as head count ratio (HCR) and poverty gap (PGAP). HCR and PGAP are based on the Foster-Greer-Thorbecke indicator \citep{foster_class_1984}:
\begin{gather*}
   F_d(\alpha, t) = \displaystyle \frac 1 {N_d} \sum_{i=1}^{N_d}\left( \frac{t - y_{di}}{t} \right)^\alpha I (y_{di} \le t), 
   \hspace{1cm}\alpha = 0, 1, 2,
\end{gather*}
where $t$ is the poverty line (60\% of median income), $y_{di}$ is the income for the $i$-th household in area $d$ and $I(\cdot)$ is the indicator function. If $\alpha = 0$, then $F_d$ is the HCR – in other words the proportion of households below the poverty line in area $d$. $F_d$ quantifies poverty intensity (PGAP) when $\alpha = 1$, i.e. it measures by how much poor people are below the poverty line on average. $\alpha = 2$ defines poverty severity, which will not be considered in this paper. After discussing the problems with income data and defining the poverty indicators of interest, the next step is to define how income data can be modelled. 




