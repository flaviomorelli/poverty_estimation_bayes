\chapter{Income Data and its Challenges}

Before discussing the Bayesian workflow, it is necessary to analyze the problem at the core of the present paper: how to adequately model income data, which is the basis for estimating poverty indicators.
The first section presents reasons why modelling income data is challenging.
It also discusses available modelling possibilities are their drawbacks briefly.
In the second section, the data set from the Mexican state of Guerrero used throughout this paper is introduced together with the relevant variables.
The survey design of the Mexican data is summarized in the third section.
While it is uncommon to explicitly discussing the survey design in the SAE literature (e.g., \cite{rojas_perilla_data_2020} AND OTHERS), it is useful for certain modelling decisions, as will become clear in chapter 4.
The last section formulates simulation scenarios that capture the main difficulties of income data, based on the scenarios in \cite{rojas_perilla_data_2020}.
The simulations have the advantage that the true data generating process is known are used to check the results from different models without having to use the data multiple times.

\section{Difficulties of modelling income data}
\label{ch:difficulties}
The estimation of poverty indicators poses two main challenges.
First, indicators such as the head count ratio and the poverty gap are non-linear transforms of income.
This makes it hard to model them directly as a dependent variable in a regression, as it is necessary to first model income.
This leads to the second challenge.
Empirically, income is characterized by a right-skewed, unimodel and leptokurtic distribution:
most people earn an average or lower-than-average income, while only a few have very high incomes.
Using a linear model as in models \ref{eq:hb_rao} or \ref{eq:mod_hb} naively will lead to poor results.
There are two possible ways to deal with this characteristic shape of income data.
First, a GLM-regression with a skewed distribution (e.g., gamma, skew-normal, lognormal, etc.) can be used to capture the main features of income.
Second, data-driven transformation can be applied to make income more symmetric or even closer to a normal distribution, e.g., in \cite{rojas_perilla_data_2020}.

Each approach presents its own challenges.
Choosing a skewed likelihood for a GLM model is all but straightforward.
First, common skewed distributions present the researcher with clear restrictions:
the skew-normal distribution has a maximum skewness of 1 and allows negative values, the lognormal implies that its logarithm must follow a normal distribution, the gamma likelihood assumes a fixed ratio between the expected value and the variance, exponential or Pareto distributions have the mode at the minimum of the support.
This problem might in principle be alleviated by choosing a distribution with more parameters that allows for more flexibility such as a generalized beta distribution.
However, this additional flexibility comes at the cost of interpretability and ease of parametrization.
In cases where the use of a highly complex likelihood distribution seems necessary, the question arises whether it might be better to use a non-parametric method instead.
Moreover, skewed likelihoods can cause problems for the sampling algorithm. HMC can have trouble with the heavy tail to the right of the distribution mode, which can increase the sampling time considerably \cite{betancourt_conceptual_2017}.

On the other hand, the approach using data-driven transformations has to choose an appropriate transformation and consider the uncertainty generated by the estimation of transformation parameters.
It is not guaranteed that after the transformation the dependent variable will display more desirable properties such as symmetry or closeness to a normal distribution.
Moreover, the complexity of backtransformation functions vary depending on the chosen transformation.
For a log or log-shift transformation, the backtransformation can be done simply with the exponential function.
Nevertheless, more complex and piecewise transformation such as the Box-Cox have less straightforward backtransformation functions.
Finally, due to backtransformation it is usually not clear what the impact of the model parameters is in the original scale.
For example, a normal distribution in the transformed scale with parameters $\mu$ and $\sigma$ becomes a lognormal distribution when backtransforming with the exponential function.
In the backtransformed scale, the mean is a function of both $\mu$ and $\sigma$ given by $e^{\mu + \frac{\sigma^2}{2}}$.
However, if the likelihood in the transformed scale is a Student's $t$-distribution and the Box-Cox transformation is used, it is difficult to know the exact effect of the likelihood parameters $\mu, \sigma$ and $\nu$ on the mean in the backtransformed scale.

There is a last difficulty related to both the skewed likelihood and the data-driven transformations approach.
Choosing a likelihood or transformation does not only entail an assumption on the dependent variable, but also on the type of model.
A skew-normal distribution assumes an additive model in the original scale, as the mean can be parametrized directly in the original scale. On the other hand, a gamma distribution with the commonly used log-link implies a multiplicative model in the original scale.
Similarly, a transformation like a log-shift with an additive likelihood in the transformed scale (e.g., normal distribution) causes the covariates to have a multiplicative effect in the backtransformed scale.
In practice, it is not possible to know whether the true data generating process is additive or multiplicative so it is necessary to check the models against both additive and multiplicative scenarios.

To explore how income can be modelled in a Bayesian context, data from the Mexican state of Guerrero as well as simulated scenarios are used to develop and showcase the model developed in the next chapter.
In the next section, the Mexican data set is introduced.

%(EVTL SPÄTERA common fix is to transform the data with the natural logarithm which is easy to backtransform to the original scale with the exponential function. While this tends to make the distribution more symmetric, there might be some skewness left. Another problem is that the kurtosis of the log-transformed income rarely matches the kurtosis of a normal distribution.
%Another approach is to do a log or log-shift transformation of the outcome variable and model it with a symmetric distribution such as a Normal, Student-t, Logistic or Cauchy.
%The log-shift transformation has the advantage of being mathematically simpler than power transformations and it is already widely used for income data in fields such as econometrics. While there might be some skewness left after taking the logarithm, this can still be minimized by choosing an appropriate shift term. Moreover, the shift term eliminates values that are close to zero which can be a problem in the logarithmic scale, since $\underset{x \rightarrow 0}\lim \log(x) = -\infty$. With almost no skewness left after the log-shift transformation, the main challenge is to model the excess kurtosis of the variable. Therefore, it is reasonable to use a distribution that allows for fat tails. This paper focuses on the second approach and explores how to best model the excess kurtosis of income in the log-scale.)


\section{Data from the Mexican state of Guerrero}
\label{ch:mexican_data}
To explore the estimation of poverty indicators for small areas, this paper uses the 2010 Household Income and Expenditure Survey (\textit{Encuesta Nacional de Ingresos y Gastos de los Hogares – ENIGH}) and the 2010 National Population and Housing Census from Mexico by the National Institute of Statistics and Geography (INEGI).
Specifically, both data sets will be used to estimate the head count ratio (HCR) and poverty gap (PGAP) presented in chapter \ref{ch:indicators}.
The state of Guerrero is divided into 81 municipalities of which 40 are in-sample and 41 out-of-sample in the income survey.
The census contains 148083 observations after cleaning the data.
Income data is provided at the household level and there are 1801 households in the income survey.
The Acapulco municipality contains 511 households – is almost a third of the survey sample, and the smallest municipality contains only 13 households.
The median sample size at the municipality level is 26.5.
Most of the state's population consists of impoverished Indians and mestizos.
However, there are important tourist destinations in the municipalities of Acapulco and Zihuatanejo on the pacific coast and also in Taxco de Alarcón in the highlands.
Guerrero's economy is based mainly on the primary sector and two fifths of the population live in rural areas \citep{encyclopaedia_britannica_guerrero_2019}.
\begin{table}[t]
    \caption{Variables related to head of household.}
    \centering
    \begin{tabular}{ l | m{8cm} | l }
        \textbf{Variable} & \textbf{Description} & \textbf{Source} \\
        \hline
        Occupation type & Occupation in the primary,
        secondary, tertiary sector or not employed
        & \code{jsector}\\
        Gender & male or female & \code{jsexo}\\
        Work experience & Years of work experience & \code{jexp}\\
        Age & Age of head of household  & \code{jedad}\\
    \end{tabular}
    \label{tab:head_household}
\end{table}


In an applied setting, the variables that can actually be chosen is limited by a number of factors.
For SAE methods, it is necessary to have auxiliary data to borrow strength and improve the estimations, which is only possible if both the main and the auxiliary data sources contain the same variables.
Moreover, the amount of missings in a variable can severly undermine its usefulness for prediction.
A large number of missings (20\% or more) in one or multiple variables can lead to a high number of observations being discarded.
While imputation is possible in principle, it is not straightforward to take the clustered and stratified structure of survey data into account in the imputation process.
Finally, some variables include missing values due to structural reasons.
For example, if the head of household is single then there will be missing values in variables that concern the partner.
Imputation in such cases is unrealistic.
Therefore, variables with a low amount of missings (<5 \%) both in the survey and in the census are chosen as candidates for the final model.

\begin{table}[t]
    \caption{Variables related to household demographics.}
    \centering
    \begin{tabular}{ l | m{7cm} | l }
        \textbf{Variable} & \textbf{Description} & \textbf{Source} \\
        \hline
        Minors under 16 & Presence of minors under 16 years old in household
        & \code{id\_men}\\
        Percentage of women & Percentage of women in household & \code{muj\_hog / tam\_hog}\\
        Literacy & Percentage of literate memebers of household & \code{nalfab / tam\_hog}\\
        Indigenous population & Presence of indigenous population in household  & \code{pob\_ind}\\
        Geography & Household in urban or rural area  & \code{rururb}\\
    \end{tabular}
    \label{tab:demo_household}
\end{table}

The income variable $y_{di}$ for a given municipality and individual corresponds to \code{icptc} in the survey and it is not available in the census.
\code{icptc} measures equivalized total household per capita income in Mexican pesos and is used as a proxy for the living standard.
To generate high-quality predictions, variables present in both data sets that are plausibly related to income are used as regressors.
These variables are grouped into three categories.
Variables related to head of household (Table \ref{tab:head_household}\footnote{For practical purposes, the \textit{not employed} category summarizes both the unemployed and individuals out of the labor force.}) are likely to have high predictive power, because the head of household is usually the main breadwinner of the household and his/her situation has a large impact on the economic situation of the whole household.
Additionally, variables related to household demographics (Table \ref{tab:demo_household}) are relevant, because they provide information about socio-demographic patterns that impact household income – e.g., rural areas tend to have lower income than rural areas, or indigenous people are more likely to be marginalized in former colonies.
Finally, variables about the economic situation of the household (Table \ref{tab:economic_household}), represent economic circumstances, which reflect household income and wealth more directly\footnote{In the tables, code notation is used for some variables. The symbol \code{/} indicates division and \code{||} indicates the Boolean OR. The variable \code{tam\_hog} is the number of members in the household and is used to compute certain quantities per capita to make them more comparable among households of different sizes.}.
These variable are only a starting point and a more careful variable selection is done in section \ref{ch:varsel}.



\begin{table}[t]
    \caption{Variables related to economic situation.}
    \centering
    \begin{tabular}{ m{3.4cm} | m{7cm} | l }
        \textbf{Variable} & \textbf{Description} & \textbf{Source} \\
        \hline
        Working members & Percentage of working members of household
        & \code{pcocup}\\
        Income-receiving members & Percentage of members who receive an income & \code{pcpering}\\
        Unusual work & Presence of child or senior work in household& \code{trabinf || trabadulmay}\\
        External income & Household receives remittances or financial help from other households  & \code{remesas || ayuotr}\\
        Communication goods & Number of communication goods per capita in household & \code{actcom / tam\_hog}\\
        General goods & Number of goods per capita in household  & \code{actcom / tam\_hog}\\
    \end{tabular}
    \label{tab:economic_household}
\end{table}

There is a last group of binary indicator variables that are based on the presence of certain structural disadvantages in the household concerning four different areas (Table \ref{tab:disadvantages}): education, health care, housing quality and access to public utilities.
These variables are not used as predictors in the regression, but they will be considered in chapter \ref{ch:raneff} when discussing whether it is possible to have an alternative definition of the random effect in the model.

\begin{table}[h]
    \caption{Indicators of structural disadvantages.}
    \centering
    \begin{tabular}{ l | m{8cm} | l }
        \textbf{Variable} & \textbf{Description} & \textbf{Source} \\
        \hline
        Education & Adequate access to education
        & \code{ic\_rezedu}\\
        Health care & Adequate access to health care & \code{ic\_asalud}\\
        Housing quality & Adequate housing quality & \code{ic\_cv}\\
        Public utilities & Adequate acces to public utilities (electricity, running water, sewer system)  & \code{ic\_sbv}\\
    \end{tabular}
    \label{tab:disadvantages}
\end{table}

%%%

\section{Introduction to the survey design}
\label{ch:design}

Before presenting the methodology used in this paper, it is crucial to have a rough understanding of the survey design of the data.
The following contains a high-level summary of this topic and more details can be found in \cite{inegi_modulo_2011}, which will become relevant in section \ref{ch:raneff}.
The MCS module of the ENIGH has a stratified, two-stage, clustered survey design.
The main unit in the survey design is called the primary survey unit (PSU), which is a cluster of households.
The exact definition of the context varies depending on the context:
while in an urban setting the PSU is defined as a grouping of street blocks, in a rural context such a definition is not possible.
In general, each PSU can contain between 80 and 300 households.

The stratification is done in multiple stages.
The first stage consists of four strata based on socio-economic indicators taken from the census.\footnote{A complete list of the indicators can be found in the appendix of the MCS documentation \citep{inegi_modulo_2011}.} These four strata contain all PSUs in the country.
The second stage corresponds to a geographic stratification.
In each federal state, PSUs are stratified according to whether they are in a rural, urban or highly urban area (\textit{ámbito}) with even finer groups inside each one of these categories (\textit{zona}).
Note that the stratification \textit{does not} consider any municipal divisions.
This fact will play a key role in chapter \ref{ch:raneff}.

Two-stage sampling indicates that inside each stratum multiple PSUs are sampled and then, in a second stage, a fixed number of households inside each selected PSU is sampled.
This sampling strategy implies that the households are clustered depending on the PSU they are sampled from.
There might be deviations from the sampling scheme presented in order to guarantee that certain global properties of the whole sample still hold – e.g., a balanced ratio between the number of women and men – or that there is enough differentiation in a given area.
The effect of nonresponse is taken into account in the survey weights.
However, in this paper the weights are not taken into account when calculating the final poverty indicators.\footnote{The reason for this is that common frequentist SAE packages in \code{R}, which will be used to compare the EBP to the Bayesian model, do not allow for the inclusion of survey weights in the estimation procedure.}

While the clustered structure arising from two-stage sampling can be accounted for when building the model, this paper focuses instead on how to integrate the stratification into the model.
Thus, the adequate consideration of the clustered structure in the model is left to future research.
The last section of this chapter introduces simulation scenarios that capture the main features of income data.




\section{Simulation scenarios}
\label{ch:simulations}
Working with simulations is one key step of iterative model improvement in a Bayesian workflow \citep{gelman_bayesian_2020}.
Testing models against synthetic data allows the researcher to check and understand potential problems with the methods.
While a model that works well with simulated data is not guaranteed to work well with real-world data,
a model that does not work with simulated data is certain to fail in a real application.
This will become clearer in the next chapter on Bayesian workflow.

As described in section \ref{ch:difficulties}, income data is characterized by a unimodal, right-skewed and leptokurtic distribution.
To mimic this characteristics, three simulation scenarios based on \cite{rojas_perilla_data_2020} are proposed.
The first one – the \textit{log-scale} scenario – is defined so that the logarithm of simulated income is roughly normal:
\begin{equation}
    \begin{split}
        u_d & \sim \mathcal N(0, 0.4), ~~ d = 1,...,D,\\
        \varepsilon_{di} & \sim \mathcal{N}(0, 0.3), i = 1,...,N,\\
        \mu_{dk} & \sim \mathcal{U}(2, 3), k = 1,...,K,\\
        \Sigma_{mn} & = \begin{cases} 1, ~~ m = n,~~m = 1,...,K, n = 1,...,K, \\ \rho,  ~~ \text{otherwise},  \end{cases}
            \\
        \boldsymbol x_{di}  &\sim \mathcal N (\boldsymbol \mu_{d}, \Sigma) ,
            ~~ \boldsymbol \mu_{d} = (\mu_{d1}, ..., \mu_{dK}),\\
        y_{di} & = \exp(5 + 0.1 \cdot \boldsymbol x_{di}  + u_d + \varepsilon_{di}),\\
    \end{split}
    \label{eq:log_scenario}
\end{equation}
where $K$ is the number of regressors, $D$ is the number of domains, and $N = N_1 + ... + N_D$ is the total number of observations.
$\Sigma$ is the covariance matrix and it assumes that all covariates have unit variance. $\rho$ controls the correlation between independent variables and is set to 0.2.
A certain varation among areas is achieved through the vector $\boldsymbol \mu_d$, which contains the means for all covariates in area $d$.
Thus, some covariates are consistently higher or lower in a given area, creating a clear profile for each area.
Because the focus is on prediction, the regression coefficients are fixed and equal for all covariates (e.g., 0.1 in the log-scale scenario), but this is only done for simplicity.
There are two additional scenarios – the \textit{Pareto} and the \textit{GB2}.
The GB2 scenario can be formulated as
\begin{equation}
    \begin{split}
        u_d & \sim \mathcal N(0, 500), ~~ d = 1,...,D,\\
        \varepsilon_{di} & \sim \mathcal{GB}2(2.5, 18, 1.46, 1700), i = 1,...,N,\\
        \mu_{dk} & \sim \mathcal{U}(-1, 1), k = 1,...,K,\\
        \boldsymbol x_{di}  &\sim  \mathcal N  (\boldsymbol \mu_{d}, \Sigma) ,
        ~~ \boldsymbol \mu_{d} = (\mu_{d1}, ..., \mu_{dK}),\\
        \tilde{\varepsilon}_{di} & = \varepsilon_{di} - \bar \varepsilon,  \\
        y_{di} & = 9000 - 250 \cdot \boldsymbol x_{di} + u_d + \tilde \varepsilon_{di}, \\
    \end{split}
    \label{eq:gb2_scenario}
\end{equation}
where $\mathcal{GB}2$ is the generalized beta distribution of the second kind with four parameters usually referred to as $a, b, p, q$. $\Sigma$ is defined as in scenario \ref{eq:log_scenario} with $\rho = 0.2$. Note that $\varepsilon_{di}$ needs to be centered, as they have a non-zero mean. The third and last scenario has a Pareto error term:
\begin{equation}
    \begin{split}
        u_d & \sim \mathcal N(0, 500), ~~ d = 1,...,D,\\
        \varepsilon_{di} & \sim \text{Pareto}(3, 2000), i = 1,...,N,\\
        \mu_{dk} & \sim \mathcal{U}(-3, 3), k = 1,...,K,\\
        \boldsymbol x_{di}  &\sim  \mathcal N (\boldsymbol \mu_{d}, \Sigma) ,
        ~~ \boldsymbol \mu_{d} = (\mu_{d1}, ..., \mu_{dK}),\\
        \tilde{\varepsilon}_{di} & = \varepsilon_{di} - \bar \varepsilon,  \\
        y_{di} & = 12000 - 350 \cdot \boldsymbol x_{di}    + u_d + \tilde \varepsilon_{di}.\\
    \end{split}
    \label{eq:pareto_scenario}
\end{equation}
Here again, $\tilde \varepsilon_{di}$ are the centered unit-level residuals.
Note that all fixed parameters in the three scenarios (e.g., the standard deviation of the area-level random effect $u_d$) are chosen so that the simulations are in a realistic range in line with the income variable \code{icptc} from the Mexican survey.
This means roughly that no simulation should be above the tens of thousands.

The three scenarios provide a variety of characteristics against which to test the models in the workflow.
First, the log-scale scenario is multiplicative, while the GB2 and Pareto scenarios are additive.
This is useful to check whether a model with a multiplicative likelihood also can approximate a model whose generating process is additive, and viceversa.
Moreover, the logarithmic transformation of $y_{di}$ should be roughly normal, while the GB2 and Pareto scenarios should display a higher excess kurtosis after a logarithmic transform, which is often observed in real-world income data.

Nevertheless, there are two main limitations with the simulation scenarios.
First, the area-level intercepts $u_d$ are assumed to be independent, by defining the standard deviation as a constant.
This is unrealistic, as there are similarities and differences between areas that are likely to manifest themselves as correlations.
A clear example would be two neighboring geographic areas with highly intertwined economies.
Second, some covariates might have different effect sizes depending on the area.
For example, a higher level of education could have a much stronger effect on income in an urban region, where there is a higher demand for specialized, well-paying jobs.
In contrast, most jobs in a rural region are less likely to require high skills and therefore offer a lower pay.
Here, income does not only depend on the quality of labor supply, but also on the job market demands in a gives area.
For the present paper, these two limitations are acceptable.
On the one hand, there are many ways of including spatial correlation in the simulation scenarios (see section \ref{ch:area_corr}) and it is not clear, whether a decision for one type of correlation might turn out to be unlike the spatial dependencis present in real-world data.
On the other hand, random slopes are not considered in the Bayesian workflow of the next chapter.
Extensions to the simulation scenarios such as spatial correlation and random slopes might play an important role in developing a model for poverty estimation, but this question is left for further research.

