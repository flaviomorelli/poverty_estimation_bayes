\chapter{Income Data and its Challenges}

The estimation of poverty indicators poses two main challenges.
First, indicators such as the head count ratio and the poverty gap are non-linear transforms of income.
This makes it hard to model them directly as a dependent variable in a regression, as it is necessary to first model income.
This leads to the second challenge.
Empirically, income is characterized by a right-skewed, unimodel and leptokurtic distribution.
Using a linear regression naively will lead to poor results.
There are two possible ways to deal with this characteristic shape.
First, a GLM regression with a skewed distribution (e.g., gamma, skew-normal, lognormal, etc.) can be used to capture the main features of income.
Second, data-driven transformation can be applied to make income more symmetric or even closer to a normal distribution.
To explore how income can be modelled in a Bayesian context, data from the Mexican state of Guerrero as well as simulated scenarios are used to develop and showcase the model developed in the next chapter.
In the next section, the Mexican data set is introduced.


\section{Mexican state of Guerrero}
This paper uses the 2010 Household Income and Expenditure Survey (\textit{Encuesta Nacional de Ingresos y Gastos de los Hogares – ENIGH}) and the 2010 National Population and Housing Census from Mexico provided by the National Institute of Statistics and Geography (INEGI) as a basis for the analysis.
Based on these two data sets, the objective is to estimate the head count ratio and poverty gap described in section 3.1 for the state of Guerrero.
This state is divided into 81 municipalities of which 40 are in-sample and 41 out-of-sample in the income survey.
Income data is provided at the household level and there are 1801 households in the income survey.
The median sample size at the municipality level is 26.5.
The Acapulco municipality contains 511 households which is almost a third of the whole sample.
The smallest sample for a municipality is 13 households.
Figure \ref{fig:guerrero} gives an overview of which municipalities are in-sample.
The census contains 152000 observations after cleaning the data.
Guerrero's economy is based mainly on the primary sector and two-fifths of the population lives in rural areas.
While most of the state's population consists of impoverished Indians and mestizos, there are important tourist destinations in the municipalities of Acapulco and Zihuatanejo on the pacific coast and also in Taxco de Alarcón in the highlands \citep{encyclopaedia_britannica_guerrero_2019}.
\begin{table}[t]
    \caption{Variables related to head of household}
    \centering
    \begin{tabular}{ l | m{8cm} | l }
        \textbf{Variable} & \textbf{Description} & \textbf{Source} \\
        \hline
        Occupation type & Occupation in the primary,
        secondary, tertiary sector or not employed
        & \code{jsector}\\
        Gender & male or female & \code{jsexo}\\
        Work experience & Years of work experience & \code{jexp}\\
        Age & Age of head of household  & \code{jedad}\\
    \end{tabular}
    \label{tab:head_household}
\end{table}

\section{Variables}

In an applied setting, the variables that can actually be chosen is limited by a number of factors.
For SAE methods, it is necessary to have auxiliary data to borrow strength and improve the estimations.
However, this is only possible if both the main and auxiliary data sources contain the same variables.
Moreover, the amount of missings in a variable can severly undermine its usefulness for prediction.
A large number of missings (20\% or more) in one or multiple variables can lead to a high number of observations being discarded.
While imputation is possible in principle, it might be challenging to take the clustered and stratified structure of survey data into account in the imputation process.
Finally, some variables include missing values by necessity.
For example, if the head of household is single then there will be missing values in variables that concern the partner.
Imputation in such cases seems unrealistic.
Therefore, variables with a low amount of missings (<5 \%) both in the survey and in the census are chosen as candidates for the final model.

The income variable $y_{di}$ for a given municipality and individual corresponds to \code{icptc} in the survey and it is not available in the census.
\code{icptc} measures total household per capita income in Mexican pesos and is used to approximate the standard of living.
As the predictive power of the model depends on the selected regressors, it is necessary to do careful variable selection.

% In the Bayesian context, variable selection can be done with PSIS-LOO which measures predictive power like information criteria such as the AIC.
% Moreover, current research in development and poverty economics can indicate which variables are reasonable to include in the model.
% However, the focus of this paper is not variable selection.

The first group of variables relate to the head of household.
This individual is usually the main breadwinner of the household and socio-economic information about him her are assumed to be predictive of the equalized income per capita:

Note that the \textit{not employed} category summarizes both the unemployed and individuals out of the labor force.
The next group of variables concern the demographic composition of the household. Explanation:
\begin{table}[h]
    \caption{Variables related to household demographics}
    \centering
    \begin{tabular}{ l | m{8cm} | l }
        \textbf{Variable} & \textbf{Description} & \textbf{Source} \\
        \hline
        Minors under 16 & Presence of minors under 16 years old in household
        & \code{id\_men}\\
        Percentage of women & Percentage of women in household & \code{muj\_hog / tam\_hog}\\
        Literacy & Percentage of literate memebers of household & \code{nalfab / tam\_hog}\\
        Indigenous population & Presence of indigenous population in household  & \code{pob\_ind}\\
        Geography & Household in urban or rural area  & \code{rururb}\\
    \end{tabular}
    \label{tab:demo_household}
\end{table}
The third and last group of variables describe the economics situation of the household.
\begin{table}[h]
    \caption{Variables related to economic situation}
    \centering
    \begin{tabular}{ l | m{7cm} | l }
        \textbf{Variable} & \textbf{Description} & \textbf{Source} \\
        \hline
        Working members & Percentage of working members of household
        & \code{pcocup}\\
        Income-receiving members & Percentage of members who receive an income & \code{pcpering}\\
        Unusual work & Presence of child or senior work in household& \code{trabinf || trabadulmay}\\
        External income & Household receives remittances or financial help from other households  & \code{remesas || ayuotr}\\
        Communication goods & Number of communication goods per capita in household & \code{actcom / tam\_hog}\\
        General goods & Number of goods per capita in household  & \code{actcom / tam\_hog}\\
    \end{tabular}
    \label{tab:economic_household}
\end{table}

In addition, there are additional indicator variables that will be used when considering an alternative specification for the random effect.
\begin{table}[h]
    \caption{Variables related to head of household}
    \centering
    \begin{tabular}{ l | m{8cm} | l }
        \textbf{Variable} & \textbf{Description} & \textbf{Source} \\
        \hline
        Education & Adequate access to education
        & \code{ic\_rezedu}\\
        Health care & Adequate access to health care & \code{ic\_asalud}\\
        Housing quality & Adequate housing quality & \code{ic\_cv}\\
        Public utilities & Adequate acces to public utilities (electricity, running water, sewer system)  & \code{ic\_sbv}\\
    \end{tabular}
    \label{tab:disadvantages}
\end{table}

\section{Survey Design}

Before presenting the methodology used in this paper, it is crucial to have a rough understanding of the survey design of the data.
The following will be a very high-level summary of this topic.
The MCS module of the ENIGH has a stratified, two-stage, clustered survey designed (source).
The main unit in the survey design is called the primary survey unit or UPM, which is a cluster of households.
The exact definition of the context varies depending on the context.
While in an urban setting the UPM is defined as a grouping of street blocks, in a rural context this definition is not feasible.
In general, each UPM can contain between 80 and 300 households.

The stratification is done in multiple stages.
The first stage consists of four strata of all UPMs in the country based on socio-economic indicators based on the census.
A complete list of the indicators can be found in the appendix of MCS (Source).
The second stage corresponds to a geographic stratification.
In each federal state, UPMs are stratified according to whether they are in a rural, urban or highly urabn area (\textit{ámbito}) with even finer groups inside each one of these categories (\textit{zona}).
Note that the stratification \textit{does not} consider municipal divisions. This will be further discussed in chapter XY.

Two-stage sampling refers to the fact that inside each stratum a number of UPMs is sampled first and then, in a second stage, a fixed number of households inside the selected UPMs is sampled.
There might be deviations from the sampling scheme presented in order to guarantee that certain global properties of the whole sample still hold – e.g., a balanced ratio between the number of women and man –, or that there is enough differentiation in a given area.
The effect of nonresponse is taken into account in the survey weights. Further details on the survey design can be found in the documentation (SOURCE, Spanish only).

\section{Simulation scenarios}

Working with simulated is one key step of Bayesian workflow (WORKFLOW PAPER).
Testing models against fake data allows the researcher to check understand potential problems with the methods.
While a model that works well with simulated data is not guaranteed to work well with real-world data,
a model that does not work with simulated data is certain to fail in a real application.

Income data is characterized by being unimodal, right-skewed and leptokurtic.
To mimic this characteristics, three simulation scenarios based on \cite{rojas_perilla_data_2020} are proposed.
The first one – the \textit{log-scale} scenario – is defined so that the logarithm of simulated income is roughly normal.
The proposed methodology should not have difficulties with this scenario, which is defined as follows:
\begin{equation}
    \begin{split}
        u_d & \sim \mathcal N(0, 0.4), ~~ d = 1,...,D,\\
        \varepsilon_{di} & \sim \mathcal{N}(0, 0.3), i = 1,...,N,\\
        \mu_{dk} & \sim \mathcal{U}(2, 3), k = 1,...,K,\\
        \Sigma_{mn} = &\begin{cases} 1, ~~ m = n \\ \rho,  ~~ \text{otherwise}  \end{cases},
            m = 1,...,K, n = 1,...,K,\\
        \boldsymbol x_{di}  &\sim \text{MvNormal} (\boldsymbol \mu_{d}, \Sigma) ,
            ~~ \boldsymbol \mu_{d} = (\mu_{d1}, ..., \mu_{dK}),\\
        \boldsymbol\beta & =  \mathbf{1}_k \cdot 0.1,\\
        y_{di} & = \exp(5 + \boldsymbol x_{di}' \boldsymbol \beta    + u_d + \varepsilon_{di}),\\
    \end{split}
    \label{eq:log_scenario}
\end{equation}
where $K$ is the number of regressors, $D$ is the number of domains, and $N = N_1 + ... + N_D$ is the total number of observations.
$\rho$ controls the correlation between independent variables and is set to 0.2.
In addition, there are two scenarios that are more challenging due to their heavy-tails in the logarithmic scales – the \textit{Pareto} and the \textit{GB2}.
The higher excess kurtosis after a logarithmic transform mirrors a situation often encountered with empirical data. The Pareto scenario usually has the heaviest tails of all three scenarios. The GB2 scenarios can be formulated as
\begin{equation}
    \begin{split}
        u_d & \sim \mathcal N(0, 500), ~~ d = 1,...,D,\\
        \varepsilon_{di} & \sim \mathcal{GB}2(2.5, 18, 1.46, 1700), i = 1,...,N,\\
        \mu_{dk} & \sim \mathcal{U}(-1, 1), k = 1,...,K,\\
        \boldsymbol x_{di}  &\sim \text{MvNormal} (\boldsymbol \mu_{d}, \Sigma) ,
        ~~ \boldsymbol \mu_{d} = (\mu_{d1}, ..., \mu_{dK}),\\
        \boldsymbol\beta & =  \mathbf{1}_k \cdot 250,\\
        \tilde{\varepsilon}_{di} & = \varepsilon_{di} - \bar \varepsilon,  \\
        y_{di} & = 9000 - \boldsymbol x_{di}' \boldsymbol \beta + u_d + \tilde \varepsilon_{di}, \\
    \end{split}
    \label{eq:gb2_scenario}
\end{equation}
where $\mathcal{GB}2$ is the generalized beta distribution of the second kind with four parameters usually referred to as $a, b, p, q$. $\Sigma$ is defined as in scenario \ref{eq:log_scenario} with $\rho = 0.2$. Note that $\varepsilon_{di}$ need to be centered, as they have a non-zero mean. The third and last scenario has a Pareto error term:
\begin{equation}
    \begin{split}
        u_d & \sim \mathcal N(0, 500), ~~ d = 1,...,D,\\
        \varepsilon_{di} & \sim \text{Pareto}(3, 2000), i = 1,...,N,\\
        \mu_{dk} & \sim \mathcal{U}(-3, 3), k = 1,...,K,\\
        \boldsymbol x_{di}  &\sim \text{MvNormal} (\boldsymbol \mu_{d}, \Sigma) ,
        ~~ \boldsymbol \mu_{d} = (\mu_{d1}, ..., \mu_{dK}),\\
        \boldsymbol\beta & =  \mathbf{1}_k \cdot 350,\\
        \tilde{\varepsilon}_{di} & = \varepsilon_{di} - \bar \varepsilon,  \\
        y_{di} & = 12000 - \boldsymbol x_{di}' \boldsymbol \beta    + u_d + \tilde \varepsilon_{di}.\\
    \end{split}
    \label{eq:pareto_scenario}
\end{equation}
Here again, $\tilde \varepsilon_{di}$ are the centered unit-level residuals.

\vspace{-0.4 cm}