\section{Comparison of Bayesian models with stacking weights}

Model selection is an uncertain procedure, especially when there are multiple plausible models that are consistent with the data \citep{gelman_bayesian_2020}.
While in section \ref{ch:area_corr} the SAR model was selected as the final model, it was also clear in Table \ref{tab:lkj_sar_base} that the model was indistinguishable from or only marginally better than the other models.
Stacking, which was discussed in section \ref{ch:bayesian_evaluation}, provides a method to combine predictions from multiple heterogeneous models.
Weights are calculated jointly for all models based on their elpd$_{\text{loo}}$, which can be used to estimate
a weighted average of the predictions from the different models.
This paper does not investigate whether stacking provides better results than a single model.
However, the weights (shown in Table \ref{tab:stacking}) provide valuable insights into the predictive power of the different models.
First, stacking is relatively insensitive to similar models.
This can be seen in the first three models, which are identical with the exception of the tightness of the prior on the skewness.
The first three models share their weights in the sense that only the weight of the third model is non-zero.
The last three entries in Table \ref{tab:stacking} correspond to the three models used to introduce area-level correlation: a base model with no correlation, the LKJ model and the SAR model.
These three models offer a different picture: all of their weights are non-zero, which indicates that they are heterogenous, or else some of the weights would be equal to zero.
Additionally, their weights (0.25, 0.31, 0.24) are in a very similar range.
This indicates that they would contribute almost equally to a weighted average of predictions.
This confirms the results in Table \ref{tab:lkj_sar_base} that showed a very similar elpd$_{\text{loo}}$ for all three models.
However, their weights are higher than the first three models, which indicates that the initial models were improved through the workflow

% latex table generated in R 4.0.5 by xtable 1.8-4 package
% Sun Aug 22 09:31:33 2021
\begin{table}[ht]
    \caption{Stacking weights of models in the workflow.}
    \begin{center}
        \begin{tabular}{rrrrrrr}
          \hline
         Mid skew. & Low skew. & High skew. & Base & LKJ & SAR \\
          \hline
         0.00 & 0.00 & 0.19 & 0.25 & 0.31 & 0.24 \\
           \hline
        \end{tabular}
    \end{center}

    \begin{adjustwidth}{45pt}{45pt}
        \footnotesize{The first three models correspond to the three different priors on skewness from section \ref{ch:log_shift}. The \textit{Base} model corresponds to the model with no area-level correlation developed in section \ref{ch:coef_var_spec}. The \textit{LKJ} and \textit{SAR} models correspond to the models with area-level correlation from section \ref{ch:area_corr}. All models used the stratified random effect discussed in section \ref{ch:raneff}.}
    \end{adjustwidth}
    \label{tab:stacking}
\end{table}

In summary, the decision to use only the SAR model is not necessarily the only possible one.
It is likely that a combined prediction with methods such as stacking will outperform the predictions from a single model, but this question is not further considered in this paper.
The estimated poverty indicators are presented in the next chapter and compared to the Box-Cox EBP from \cite{rojas_perilla_data_2020}.

