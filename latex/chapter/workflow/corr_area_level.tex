\section{Modelling correlations at the area-level}

Until now, the assumption has been that all random effects at the area-level are independent from each other, i.e., $u_d|\sigma_u \sim \mathcal N (0, \sigma_u)$.
However, this assumption seems unplausible.
For example, if the areas are defined as municipalities, it is likely that neighbouring regions have similar characteristics and are therefore correlated.
If on the other hand the areas are the strata considered in the previous section, the 16 strata that correspond to a rural area have more in common with each other than with urban strata.
Here, the independence assumption does not seem realistic either.
In the section, a new prior that captures correlations at the area level is introduced to the model.
The result is then compared to the model that was defined in the previous two sections.
Thus, the areas are not defined as the municipalities, but as the strata presented in the last section.

There are multiple options to model correlation between areas.
Approaches such as IAR, CAR and SAR (SOURCE) represent a traditional way of capturing spatial correlation. Moreover, other priors such as the Hidden Markov fields used in (GAO) offer an alternative to modelling dependencies between areas.
However, the simpler LKJ prior (SOURCE) will be used in this paper.
The resulting model will be compared with the model that assumes independence from the previous section.
A comparison between all possible priors that capture area correlation is beyond the scope of this paper and left for future research.

Let $\Sigma$ be a $D \times D$ covariance matrix, so that $u \sim \mathcal{N}(\boldsymbol{0}, \Sigma)$, where $\boldsymbol 0$ and $u$ are $D$-dimensional vectors.
Therefore, the density of $u$ can be written as:
\begin{gather*}
    p(u|\Sigma) = (2\pi)^{-\frac D 2}\det(\Sigma)^{-\frac 1 2} e^{(-\frac 1 2 u'\Sigma^{-1} u)}.
\end{gather*}
The covariance matrix $\Sigma$ can be decomposed as $\Sigma = \text{diag}(\tau)\Omega\text{diag}(\tau)$, where $\tau$ is a $D$-dimensional vector of scale factor, $\text{diag}(\tau)$ is a $D \times D$ matrix with $\tau$ as its main diagonal and $\Omega$ is a $D \times D$ correlation matrix.
Up to this point, the assumption has been that there is only one scale parameter $\sigma_u$ for all area-level effects $u_d$ and this assumption still holds.
Because $\tau_d = \sqrt{\Sigma_{d, d}}$, the decomposition is simplified to $\Sigma = \sigma_u^2 \Omega$.

The standard deviation for the random effect $\sigma_u^2$, already has the prior $Ga(2, 10)$.
However, the correlation between the single random effects is captured by defining a LKJ prior over the correlation matrix $\Omega$:
\begin{gather*}
    \Omega \sim \text{LKJ}(\eta), ~~ \eta \ge 1.
\end{gather*}
The LKJ correlation distribution implies that $p(\Sigma|\eta) \propto \det(\Sigma)^{\eta - 1}$ (STAN USER MANUAL).
Note that the determinant of $\Sigma$ increases as the correlation between components decreases:
an identity matrix has a determinant of 1, while a correlation matrix consisting only of ones (perfect correlation between components) has a determinant of 0.
For $\eta = 1$, the LKJ prior is a uniform distribution over all possible correlation matrices.
However, due to the fact that the function $f(x) = k^x$ for a fixed $k \in (0, 1)$ does not converge uniformly towards zero as $x \rightarrow \infty$, a higher $\eta$ puts more probability mass on matrices with a higher determinant, i.e., on matrices with a lower correlation between components.\footnote{In practice, it is common to use the Cholesky decomposition of $\Omega$ to avoid numerical issues when estimating the model.
For clarity, the traditional matrix specification is kept in the paper, but note that the decomposition is used in the accompanying code.}

The results from the model comparison can be seen in in Table XY.
The model that allows for correlation between areas has a slightly better performance according to PSIS-LOO than the model with no correlation at all.
However, it is important to remember that elpd$_{\text{LOO}}$ is just an approximation, because PSIS-LOO quantifies predictive power at the unit-level and not at the area-level, whereas the aim of the model is to generate prediction at the area-level.
It is likely that the difference between the predictive power of both specifications would be clearer when using leave-one-group-out cross-validation.
Unfortunately, this is still an area of active research and it is therefore beyond the scope of this paper.
In any case, the slightly better performance of the model with the LKJ priors is taken as a sign that it captures correlations between strata that improve predictions.
Eta parameter?

Figure XY shows four posterior draws for Omega as heatmaps. In the plot, it is possible to see a very clear pattern, that is consistent among draws.


In the current model specification, the random intercept was redifined so that there are no out-of-sample areas.
Nevertheless, the LKJ prior is not as useful when there are out-of-sample regions, as the correlation matrix is assumed to have as many dimensions as in the training set.
In such cases, an autoregressive or a random walk prior on $u_d$ such as in RUE/HELD 2005 or GAO et AL 2021 can be more appropriate, as it does not depend strictly on the dimensions of the correlation matrix.


\includegraphics[width=\textwidth]{./graphics/rand_intercept/corr_plot}
\includegraphics[width=\textwidth]{./graphics/rand_intercept/dens_corr_plot}






