\section{Specification of coefficient and variance priors}
\label{ch:coef_var_spec}

After deciding that the likelihood should include a shift term and selecting the subset of covariates with the best predictive power, the next step will focus on the plausibility of the priors introduced in model XY from (Morelli 2021).

Section XY showed that even with a shift term in the logarithmic transformation, the coefficients can be interpreted as the approximate percentage change of the dependent variable in the original scale.
This approximation holds mostly for coefficients between -0.3 and 0.3.
However, a change of around 30\% in the original scale for each additional unit is itself very high and this fact can be included in the prior distribution.
Moreover, note that through the use of the logarithmic transform the covariates have a multiplicative effect on the original scale.
Thus, somewhat large coefficients in the logarithmic scale can have a huge impact when backtransforming to the original scale.

There are two main ways to define priors on coefficients, either by assuming that the coefficient priors are independent or by using a joint prior such as a horseshoe prior.
In this section, two independent priors are proposed for the coefficients, which are then compared with the joint prior (ja, wirklich?). However, in this case there are relatively speaking few variables and a horseshoe prior might make more sense with more variables (?)
The comparison will be done with prior predictive checks based on the variables from the selected variables of the MCS survey.
For the independent priors, two different types of distribution can be considered, either a normal distribution or a heavy-tailed distribution such as the Student's $t$ with 3 degrees of freedom.
In this case, a distribution with heavier tails is not as desirable, because even if most of the probability mass is contained in the interval between -0.3 and 0.3, the prior would allow more extreme values.
This can be avoided by choosing a Gaussian distribution, which does not have as much probability mass on extreme values.
The coefficient prior is parametrized to have a mean of zero and a standard deviation of 0.2, which implies that the 5\% and 95\% quantiles are around -0.3 and 0.3 respectively.

Another key element of the models is how the priors are defined for the standard deviations at the unit-level ($\sigma_e$) and area-level ($\sigma_u$).
As the likelihood is a generalized Student's $t$-distribution, the parameter $\sigma_e$ cannot be interpreted directly as the standard deviation of the distribution.
Given a random variable $Y$ that follows this distribution, the variance is given by $Var(Y) = \sigma_e^2 \frac{\nu}{\nu - 2}$.
To reason more easily about the unit-level variancem a new parameter $\sigma = \sqrt{Var(Y)}$ is introduced so that $\sigma_e = \sigma \sqrt{\frac{\nu - 2}{\nu}}$.
In line with model XY, the distribution $\sigma \sim Ga(2, 0.75)$ is still used.
The effect of the prior on $\sigma$ and $\sigma_u$ will be analyzed as part of the prior predictive checks.

For the prior predictive checks, the shift term is dropped. The shift term depends on the dependent variable so it is not possible to sample from its prior distribution.
Besides, it amounts to an additive constant in the backtransformed scale ($e^{y^*} - \lambda$). The impact of the exponential function is therefore much larger than the effect of the shift term $\lambda$.
The range of Mexican income data is in the tens of thousands, which is also captured in the simulation scenarios.
Therefore, the prior predictive simulations should not be above 12 in the logarithmic scale (which corresponds to around 160000 pesos in the original scale).
Conversely, as the original data has almost not observations below 1, the values in the logarithmic scale should not go far below zero.
Finally, note that the degrees of freedom in the Student's $t$-distribution can lead to very extreme simulations when $\nu$ is low.
As a consequence, the prior for $\nu$ is changed to $Ga(2, 1)$ for the prior predictive check only to get lower values.
The minimum value for $\nu$ is 2, so that the variance of the distribution is still finite.
The assumption here is that if the prior predictive simulations are not too extreme for low degrees of freedom, they will also be in a reasonable range for high values of $\nu$, for which the likelihood has thinner tails.

The results for the starting model in XY can be seen in Figure XY.
The scale is much larger for the prior predictive draws, which is especially problematic in the logarithmic scale.

The priors for the variance parameters are tightened to $Ga(2, 10)$.
This yields more realistic results.

The results for prior predictive checks for even variances can be found in appendix XY.

\begin{figure}
    \includegraphics[width=\textwidth]{./graphics/prior_predictive_checks/prior_check_pareto_start}
    \label{fig:ppc_start}
    \caption{Prior predictive check.}
\end{figure}

\begin{figure}
    \includegraphics[width=\textwidth]{./graphics/prior_predictive_checks/prior_check_pareto_tight}
    \label{fig:ppc_tight}
    \caption{Prior predictive check.}
\end{figure}




