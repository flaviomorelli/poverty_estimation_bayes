\subsection{Alternative 2: Skewed likelihoods}
\label{ch:skewed_likelihoods}
A natural question raised by income data is whether a skewed likelihood might provide the best results.
In line with the Bayesian workflow proposed by \cite{gelman_bayesian_2020}, this paper is not limited to a single initial model based on data-driven transformations, but also explores the impact of using skewed likelihoods.
The following skewed likelihoods were taken into account: gamma with logarithmic link, gamma with
softplus link, lognormal, skew-normal and exponentially modified Gaussian (exGaussian).
This variety goes beyond the distribution shape, but it also affects the assumed relation between dependent variable and predictors.
The gamma with a log link and lognormal distributions imply a multiplicative model, whereas the rest assume that the predictors have an additive impact on the dependent variable.

\begin{figure}[h]
    \centering
    \begin{subfigure}{0.45\linewidth}
        \centering
        \includegraphics[width=\textwidth]{./graphics/skewed_likelihood/single_graphs/skewnormal_logscale}
        \caption{Skew-normal}
        \label{fig:logscale_skewnormal}
    \end{subfigure}
    \begin{subfigure}{0.45\linewidth}
        \centering
        \includegraphics[width=\textwidth]{./graphics/skewed_likelihood/single_graphs/exgaussian_logscale}
        \caption{(b) exGaussian}
        \label{fig:logscale_exgaussian}
    \end{subfigure}

    \caption[Posterior predictive check for skew-normal and exGaussian likelihoods.]{Posterior predictive check for skew-normal and exGaussian likelihoods  in the log-scale scenario. The black line is the density of the dependent variable, while the blue lines represent 100 draws from the predictive posterior distribution. Neither the skew-normal nor the exGaussian likelihood provide an adequate approximation. Moreover, the chains in the exGaussian model have trouble mixing, which is clearly seen in the two diverging densities of posterior predictive samples.}
    \label{fig:logscale_misfit}
\end{figure}

The results of these alternative models are presented in this section.
The package \code{brms} \citep{burkner_brms_2017}, which provides a user-friendly interface to \code{Stan}, is used to do a first check of the likelihoods.
This follows a principle from \cite{gelman_bayesian_2020}: when working with preliminary models, it is better to use tools that allow for quick checks.
\code{brms} already has a number of likelihoods implemented, and if modifications are needed, it is always possible to take the \code{Stan} code and modify it.
But starting coding from scratch with \code{Stan} would take longer, especially because working with different likelihoods, requires different parametrizations.

Some of this likelihoods proved to be very poor fits.
The skew-normal likelihood captures well the main characteristics of the GB2 scenario.
On the other hand, it is not adequate for the logscale scenario, which is characterized by a much higher skewness, as shown in Figure \ref{fig:logscale_skewnormal}.
This is due to the limitations of the skew-normal distribution, as it has a maximum skewness of $\pm 1$.
Moreover, in this scenario the model produces some negative predictions.
Because this likelihood is unreliable for the log-scale scenario, the model with a skew-normal likelihood is not further developed.
The exGaussian likelihood also produces negative predictions in some scenarios and it is not skewed enough for the log-scale scenario (Figure \ref{fig:logscale_exgaussian}).
The graph shows that the chains do not mix well, with two different posterior predictive distribution as a result.
Neither placing a more restrictive prior on the coefficients nor decreasing the step size of the NUTS algortihm did alleviate the convergence problem.
Nevertheless, even if the MCMC chains mix well, Figure \ref{fig:logscale_exgaussian} suggests that the exGaussian function is unlikely to be skewed enough to capture the dependent variable from the log-scale scenario.
Therefore, this version of the likelihood is also discarded.

The gamma likelihood with a logarithmic link (Figure \ref{fig:ppc_gamma_log} in appendix \ref{appendix:ppc_skewed}) performs quite well on all three scenarios.
Only in the GB2 scenario, the posterior predictive draws are somewhat flatter than the dependent variable.
While the median is captured well in all three scenarios, the IQR is clearly higher in the GB2 scenario and somewhat higher in the Pareto scenario.
The mean and the standard deviation of the dependent variable and the posterior predictive samples are quite similar, which is not a suprise, due to the parameterization of the gamma likelihood.
Note the correlation between mean and standard deviation, due to the fact that the coefficient of variation ($\sigma^2/\mu $) is always equal to the shape parameter of the gamma distribution.
More details can be found in appendix \ref{appendix:ppc_skewed}.
A gamma likelihood with a softplus link function (not shown) produced extremely small predictions, far away from the original order of magnitude and is therefore not taken into consideration.
An additional check is provided by a lognormal likelihood (Figure \ref{fig:ppc_lognormal} in appendix \ref{appendix:ppc_skewed}), which performs well in all three scenarios according to the posterior predictive check.
It is not suprising that the posterior predictive checks for the lognormal likelihood and the gamma with a logarithmic link are almost identical, due to the similarity of their shapes.

There are other skewed distributions, for which the mode is equal to the minimum.
Some examples include the exponential, the Pareto the chi-squared and the half-normal distributions.
Both the exponential and the chi-squared are special cases of the gamma distribution, which makes them less flexible as likelihood than using a plain gamma distribution.
The half normal has very thin tails and a very low skewness that does not correspond with the long tails observed empirically in income data.
Therefore, they are not taken into account as a likelihood

In summary, based on the posterior predictive checks, only the lognormal likelihood and the gamma with log link worked well for all three scenarios. The next section briefly discusses the advantages and disadvantages of the two types of initial models considered: a model with a log-shift data-driven transformation and a model with a skewed likelihood.

