\section{Specification of random effect}
\label{ch:raneff}
In applications of small area estimation methods, the random effect is usually defined as the domain for which the indicators should be calculated.
For example, \cite{rojas_perilla_data_2020} use a random effect at the municipality level in the model, as the poverty indicators of interest have to be calculated for each municipality.
However, this approach poses two problems.
First, the information contained in the survey design might not be used.
There is no guarantee that the random effect is meaningful from the perspective of how the sample was constructed.
Second, it might lead to very sparse areas with lots of out-of-sample areas.
In this section, the traditional way of defining the random effect is compared to an approach that approximates the stratification in the survey design.

Section XY included a succint introduction to the survey design of the data from the Mexican state of Guerrero.
On closer inspection, municipalities play no role whatsoever in the sampling process of the households.
Instead, the population is stratified by federal state, by geographic region (urban or rural) and also by socio-economic indicators taken from the previous census.
Changing the definition of the random intercept does not have an adverse impact on the calculation of FGT indicators.
As the data is a the unit-level and each observation has information on which municipality it belongs to, it is not necessary that the random effect coincides with the municipality.
Predictions for income at the unit-level can be generated from any model and the FGT indicator can be calculated simply by using the municipality as a grouping variable.

A major challenge is to find a structure that can be found both in the survey and the census.
While the survey includes a variable that indicates the stratum to which a given observation belongs, this cannot be matched with the stratum in the census.
Therefore, the stratification variable in the data cannot be used directly.
However, there are some variables in both the survey and the census that can be used to approximate the stratification procedure.
Besides the \code{rururb} variable that indicates whether the observation corresponds to a rural or urban area, there are four additional binary variables (Table \ref{tab:disadvantages}), which contain information of disadvantage in areas such as education, health care, housing quality and access to public utilities.
As all five variables are binary, there are $2^5 = 32$ possible combinations.
Each one of these 32 combinations is now considered as a domain that is used to define the varying intercept in the model.
How sparse? Do we still have out-of-sample areas?

Using the priors defined in the previous section, the two definitions of the

How does it fare in terms of predictive power (PSIS-LOO)?

The results from PSIS-LOO tell us that the models are almost indistinguishable and that the alternative definition of the random effect might perform slightly better.
Even if the models are not substantially different in predictive terms, there is one substantial advantage of using the alternative specification: there are no out-of-sample domains.
This reduces drastically the uncertainty in the predictions for out-of-sample municipalities.

Note that there are some limitations to this approach.
First, the stratified structure is only an approximation.
Second, there are still other dimensions that are not taken into account that might still be problematic, e.g., the clustering that comes with using primary sampling units (besser erklären).
Third, this is an example that is limited to the data from Mexico.
Nevertheless, the main insight from this section is that for unit-level data the random effect does not have to follow the level at which indicators are estimated.
Looking for alternatives specifications for the random effect might lead to better predicitive power and to less out-of-sample domains.