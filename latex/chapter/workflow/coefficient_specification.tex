\section{Specification of coefficient priors}

After deciding that the likelihood should include a shift term and selecting the subset of covariates with the best predictive power, the next step will focus on the regression coefficients.
Section XY showed that even with a shift term in the logarithmic transformation, the coefficients can be interpreted as the approximate percentage change of the dependent variable in the original scale.
This approximation holds mostly for coefficients between -0.3 and 0.3.
However, a change of around 30\% in the original scale for each additional unit is itself very high and this fact can be included in the prior distribution.
Moreover, note that through the use of the logarithmic transform the covariates have a multiplicative effect on the original scale.
Thus, somewhat large coefficients in the logarithmic scale can have a huge impact when backtransforming to the original scale.

There are two main ways to define priors on coefficients, either by assuming that the coefficient priors are independent or by using a joint prior such as a horseshoe prior.
In this section, two independent priors are proposed for the coefficients, which are then compared with the joint prior (ja, wirklich?).
The comparison will be done with prior predictive checks based on the variables from the selected variables of the MCS survey.
For the independent priors, two different distributions will be tested – the normal and and the Student's $t$ with 3 degrees of freedom.
The parameters of both distributions are defined such that the interval between -0.3 and 0.3 corresponds to
95\% of the probability mass.
The heavy tails of the Student's $t$-distribution with 3 degrees of freedom make the prior somewhat weaker as it allows larger values.

Prior predictive checks




