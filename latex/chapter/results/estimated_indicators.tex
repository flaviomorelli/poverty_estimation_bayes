\section{Estimated poverty indicators with Bayesian model and EBP}
In this section, the estimates from the HB model with the SAR prior are presented and compared with the Box-Cox EBP estimates.
Figure \ref{fig:maps} shows the HCR and estimates for both the SAR HB model and the EBP.
The EBP was estimated with the \code{emdi} \citep{kreutzmann_r_2019} package with the Wild bootstrap option.
In the HB model, the weights are included in the estimation procedure.\footnote{The weights $w_{di}$ are considered in the poverty line $t^{(s)}$ by using the weighted median and when taking the weighted mean for a given a Monte Carlo sample – i.e., $F^{(s)}_d(\alpha, t^{(s)}) = \frac 1 {\sum_{i=1}^{N_d} w_{di}} \sum_{i=1}^{N_d} w_{di}\left( \frac{t^{(s)} - \tilde y_{di}}{t^{(s)}} \right)^\alpha I (\tilde y_{di} \le t^{(s)})$}
For the Bayesian model the standard deviation is provided as defined in section \ref{ch:indicators} and the RMSE is given for the EBP.
While the RMSE and the standard deviation are related, they are not directly comparable as the RMSE also measures bias.


\begin{figure}
    \begin{subfigure}{0.49\linewidth}
        \centering
        \includegraphics[width=\textwidth]{./graphics/maps/hb_hcr_mean}
        \caption{HCR estimate (HB)}
    \end{subfigure}
    \begin{subfigure}{0.49\linewidth}
        \centering
        \includegraphics[width=\textwidth]{./graphics/maps/ebp_hcr_mean}
        \caption{HCR estimate (EBP)}
    \end{subfigure}

    \begin{subfigure}{0.49\linewidth}
        \centering
        \includegraphics[width=\textwidth]{./graphics/maps/hb_hcr_sd}
        \caption{HCR standard deviation (HB)}
    \end{subfigure}
    \begin{subfigure}{0.49\linewidth}
        \centering
        \includegraphics[width=\textwidth]{./graphics/maps/ebp_hcr_rmse}
        \caption{HCR RMSE (EBP)}
    \end{subfigure}

    \begin{subfigure}{0.49\linewidth}
        \centering
        \includegraphics[width=\textwidth]{./graphics/maps/hb_pgap_mean}
        \caption{PGAP estimate (HB)}
    \end{subfigure}
    \begin{subfigure}{0.49\linewidth}
        \centering
        \includegraphics[width=\textwidth]{./graphics/maps/ebp_pgap_mean}
        \caption{PGAP estimate (EBP)}
    \end{subfigure}

    \begin{subfigure}{0.49\linewidth}
        \centering
        \includegraphics[width=\textwidth]{./graphics/maps/hb_pgap_sd}
        \caption{PGAP standard deviation (HB)}
    \end{subfigure}
    \begin{subfigure}{0.49\linewidth}
        \centering
        \includegraphics[width=\textwidth]{./graphics/maps/ebp_pgap_rmse}
        \caption{PGAP RMSE (EBP)}
    \end{subfigure}
    \caption{Mean and uncertainty estimation for the HCR and PGAP indicators.}
    \label{fig:maps}
\end{figure}

Both the HCR and PGAP estimates are nearly identical for both the HB and EBP estimates, though the HB estimate is slightly higher in some regions.
The pattern in the estimates corresponds is consistent with the economic structure of the state.
Municipalities with a strong touristic sector such as Acapulco and Zihuatanejo on the Pacific coast or Taxco de Alarcón in the north have the lowest values for both indicators.
This is also true for the municipal seat reagion Chipalcingo de los Bravo.
On the other hand, poverty is higher in the rural southeastern municipalities near the neighboring state of Oaxaca.

The uncertainty maps (standard deviation and RMSE) show a more detailed picture.
Note that the HB uses the stratified random effect from section \ref{ch:raneff}, whereas the EBP uses the municipality as the random effect like in the original BHF specification.
For the EBP, the uncertainty maps for the HCR and PGAP display a clear pattern of out-of-sample municipalities that have a higher RMSE.
This is not the case for the HB model, as the stratified random effect specification has no out-of-sample areas.
Moreover, some municipalities have a standard deviation that is just one tenth of the corresponding standard deviation in \citep{morelli_hierarchical_2021}, which is a drastic reduction in uncertainty.
Additional maps of the coefficients of variation can be found in appendix \ref{appendix:cv_maps}.

Lastly, it is clear that the comparison of uncertainties is limited by the different specifications of the random effect in the HB and EBP models.
Unfortunately, at the moment of writing it is not possible to define a random effect that differs from the area definition with the \code{emdi} package.
Such a comparison can shed more light on the relation between frequentist and Bayesian models and is left for future research.

