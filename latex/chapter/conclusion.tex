\chapter{Conclusion}

This paper applied the Bayesian workflow from \cite{gelman_bayesian_2020} to the estimation of poverty indicators.
The workflow developed in this paper showcases how a Bayesian model can be improved iteratively in the small area estimation context and how the model can be made more explainable by using different diagnostic tools.
This explainability is particularly important for public policy decision-making.
Nevertheless, there is still an open question regarding the systematic differences between the final EBP and HB estimates.
Further research is needed to answer this question.
A simple extension to the workflow would be to check the poverty estimates after each iterative improvement.
This would allow a comparison of the HB results to other methods such as the EBP or a direct estimate after each step and identify at which point the results from the methods start diverging.
Additionally, in the present paper only point estimates were taken into account.
It is worthwhile to explore new ways of presenting the whole distribution of the estimates.

A more complete workflow would also include estimates from other types of models – e.g., with skewed likelihoods.
Moreover, due to the modular nature of Bayesian models it is possible to include benchmarking \citep{pfeffermann_new_2013} into the model itself.
For example, the weighted average of expected values for each area can be defined to match a weighted average of direct estimates.
This would guarantee that the results are in line with the unbiased direct estimator.
