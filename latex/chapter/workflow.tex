\chapter{Iterative Bayesian Modelling for Poverty Estimation}

\label{ch:workflow}

This chapter shows how to develop a Bayesian model iteratively to estimate poverty indicators.
The main ideas of Bayesian workflow according to \cite{gelman_bayesian_2020} are summarized in the first section. As the workflow is highly non-linear, the sections afterwards do not correspond exactly to single steps from the workflow.
Instead, each section is a model improvement iteration, in which many (but not necessarily all) of the steps in the workflow are followed.
The second section presents two plausible alternatives to model skewed, leptokurtic and unimodal variables such as income. GLM models with skewed likelihoods are compared with a data-driven approach similar to \cite{morelli_hierarchical_2021}.
In the third section, the regularized horseshoe prior is explained, which is then used to select variables with adequate predictive power.
The rest of the chapter focuses on the impact of priors on model perfomance.
Section four discusses how to define coefficient and variance priors with the help of prior predictive checks.
In section five, the stratification described chapter \ref{ch:design} is used to redefine the areas and consequently the random effect.
Section six explores two alternatives to model correlations at the area-level: the LKJ prior and the SAR prior.
Finally, the last section compares the models using stacking weights.


\section{An introduction to Bayesian workflow}

The idea of statistical workflow is not new.
One early example of a statistical workflow can be found in% \cite{box_science_1976}.
In the Bayesian context, \cite{gabry_visualization_2019} and \cite{betancourt_towards_2020} have referred to the idea of a Bayesian workflow.
This section summarizes the ideas behind Bayesian workflow as presented by \cite{gelman_bayesian_2020}.

While Bayesian \textit{inference} deals with the formulation and computation of (conditional) probability densities, Bayesian \textit{workflow} consists of three steps: model building, inference and model checking/improvement.
In Bayesian workflow, it is inevitable to fit a series of models iteratively.
Flawed models are a necessary step towards improving the model and finding models that are useful in practice.
At the same time, model improvement is not limited to finding the best model, but it also allows to better understand the models used; specifically, why they fail or lead to different results under certain conditions.
There are several reasons for considering a workflow and not just plain inference.
Bayesian computation is challenging and it is often necessary to iterate through simpler and alternative models, sometimes using faster but less precise approximation algorithms.
Moreover, it might not be clear ahead of time which model adequate and how it can be modified or extended.
The relation between fitted models and data can be best understood by comparing inferences from different models.
For example, it is possible to gain valuable insights by comparing a model to simpler or more complex models.
Finally, there is uncertainty associated with model choice, as different models might display diverging but realistic results for the same application.

The graphical representation of the workflow is included in Figure \ref{fig:gelman_wf}.
The workflow contains many steps, but the authors emphasize that there are some steps that might be skipped or changed depending on the data and the use case.
For this reason, \cite{gelman_bayesian_2020} argues that a workflow is more general than an example, but less clearly specified than a methodology.
While the exact steps are likely to vary depending on the specific application, a workflow provides a framework to develop statistical models.
Note that this workflow is mostly focused on data modeling.
Other steps such as data collection are not taken into account.

\begin{figure}
    \includegraphics[width=16cm]{./graphics/workflow}
    \label{fig:gelman_wf}
    \caption{Workflow from Gelman et al. (2020).}
\end{figure}


An exhaustive discussion of the whole workflow is beyond the scope of the present paper.
Instead, the focus is on how the ideas in \cite{gelman_bayesian_2020} can be applied to the estimation of poverty indicators in the small area estimation context.
Due to the non-linear nature of the workflow, the sections and subsections in the current chapter are not strictly named after single steps of the workflow.
On the contrary, each section in this chapter reflect specific aspects of a model that are investigated separately.
However, throughout the rest of the paper there are explicit references to the corresponding step and an explanation of why it is necessary at a given stage.
For completeness, the rest of this section summarizes the seven workflow steps in \cite{gelman_bayesian_2020} that are also represented in Figure \ref{fig:gelman_wf}.

\textbf{Step 1. Pick an initial model}

Usually, the starting point is to adapt an idea that already exists in the literature.
This adaptation can be done in different ways: (i) start with a simple model and add layers of complexity, (ii) simplify a complex model, so that it is more understandable or easier to fit while still delivering a similar performance, (iii) consider different starting models with diverging assumptions and follow multiple paths.
Bayesian models are highly modular, as priors and likelihood can be replaced with other distributions if necessary.
Moreover, there is flexibility regarding how parameter priors interact with each other, which allows for a high degree of model complexity.

%(HERE? )Prior predictive checks are a valuable tool when building a model, because it allows to refine the model without fitting the data multiple times.
%This is especially important when the priors should regularize the model to avoid extreme predictions.
%Additionally, fully generative model?

Here, the initial model is the HB model by \cite{molina_small_2014} with its extensions in \cite{morelli_hierarchical_2021}, where the effect of different likelihoods for log-shift transformed income was explored.
The $t$-distribution provides the best performance over many different scenarios.
However, in \cite{morelli_hierarchical_2021} the shift parameter was found through a heuristic whereas
the initial model in the present paper does full inference on the shift parameter from the start.
Additionally, an alternative group of initial models that use skewed likelihoods instead of transformation will be considered in section \ref{ch:skewed_likelihoods}.



\textbf{Step 2. Fit the model}

Section XY already discussed the two most common Bayesian estimation approaches: MCMC and variational inference.
When fitting a model, the user is confronted with decisions on which algorithm to run and under which conditions.
To make an adequate choice, it is necessary to be aware of the modeling stage.
MCMC provides the most exact approximation, which is more robust with more samples and more chains.
However, if a model was just modified or the user is at an early stage of model development, it is not efficient to fit the model the most exact algorithms and a high number of samples.
Instead, the aim is for the fit of a bad model to fail fast, as models that lead to computational problems are often not adequate.
Such problems can already be clear with just two Markov chain and a drastically lower number of sample than would be taken for final inference.
Even a check with an approximate method such as variational inference might be enough to notice problems with the model, while being drastically faster than HMC.
In this paper, variational algorithms were often used to do a first fit of the model.
While iterating through the models only 2 Markov chains were used and, depending on fitting time, the number of iterations was reduced compared to the degault in \code{Stan}.


\textbf{Step 3. Validate computation}

Diagnostics for Bayesian computation methods were already discussed in Section XY.
In the HMC context, the two main aims are to have no divergences and to have an $\hat R$ lower than 1.01.
All the models presented in this paper fulfill at least these two conditions.
However, adequate diagnostic values are a necessary but not sufficient condition for a reliable model.
To assess model quality, it has to be fitted.
However, using real data can be challenging, as there is no way to distinguish modeling issues from computational issues.
This problem can be avoided by using simulated data, ideally from different scenarios.
Section XY, already presented three scenarios used in this paper.
A model that can fit fake data, is not necessarily correct, but a model that fails when fitting simulating data will also fail with real data.
This paper deals primarily with prediction, so it is enough if samples from the posterior predictive distribution capture the main characteristics of the data.
There will be less emphasis on correcltly recovering parameters from the simulated data.

(Another approach with simulated data is simulation-based calibration.
First, model parameters are generated from the prior and used to simulate data conditional on these values.
Then the model is fitted to data and the posterior is compraed to the simulated parameters from the priors that were used to generate the data.
By repeating this procedure, it is possible to check the inference algorithms – the prior should be recovered when performing inference on data sets drawn from the prior.)

\textbf{Step 4. Address computational issues}

A model that leads to computational problems usually has some underlying modelling issues.
Usually, it is best to start with a simple model and make it more complex one step at a time, making it easier to determine what part of the model is causing problems.
On the other hand, if the model is already complex and shows signs of computational problems, then it is useful to simplify it step by step until the computation is successful.
For example, in a model with numerous groups of random intercepts or random slopes it makes sense to start with just one group and then add each additional group one step at a time, so as to know whether the estimation is working.
A common source of computational problems is prior choice.
Tightening moderately informative priors can help by pushing the sampler towards certain regoins of the parameter space.
However, this adjustment of priors should be in line with available knowledge and not only to solve fitting problems.
One example will be discussed in section XY (prior on shift parameter).

\textbf{Step 5. Evaluate and use the model}

Posterior predictive checks are a useful tool when diagnosing fit problems to the data.
This can be seen as a safeguard against misspecification.
Moreover, such checks might reveal which aspects of the data are not captured well by the model.
Cross-validation is an alternative to posterior predictive checks and has the advantage that part of the data is left out.
Thus, it is less optimistic than posterior predictive checks, which uses the data for model fitting and evaluation.
While refittig model multiple times to do cross-validation can be computationally expensive, there are efficients approximations such as PSIS-LOO.
(Check LOO-PIT in Gabry 2019)

To check how informative the data with respect to a parameter, one can compare the standard deviation of prior and posterior parameters. A higher shrinkage in uncertainty indicates that the data is more informative.

\textbf{Step 6. Modify the model}

Bayesian statistics provides a modular approach in which models can be expanded or reduced in response to new data or failures to fit the model to the data.
Mix of fitting the data and domain expertise.
How are the data linked to the underlying parameters?
How can we use additional data?
The prior is a choice of what kind of available information is integrated into the model and acts as a constraint on the fitting procedure.
There are various levels of priors from completely non-informative to highly informative.
However, the way prior information acts on this information depends on the type of parameter.
Parameters controlling central quantities like a mean are less sensitive to weak prior than scale parameters such as variance.
In turn, scale parameters are less sensitive to weak priors than shape parameters, which control the tails of a distribution.
When expanding the model with additional parameters (e.g., introducing random intercepts or random slopes), it should be considered whether the priors should be tightened to stabilize the estimates, as the amount of data has not changed.

Note that the std. deviation of a $t$-distribution varies with the df.
Joint priors?


\textbf{Step 7. Compare models}

Models are fitted many times, for multiple reasons.
It might be easier to start with simple models, before getting to a more complex model. There are often bugs in the code and in the models.
A model might be well-specified, but it could be improved by expanding it.
The priors might be only placeholders, which will be replaced at a later stage.
Visualize with multiverse comparison.
If many models provide acceptable conclusion, then

Comparing different models is always tied to a certain degree of uncertainty.
Instead of choosing the model with the best cross-validation results, using model stacking can give an insight into model differences.
Stacking combines inferences using a weighting that minimizes cross-validation error (Yao).
Roughly speaking, if a model outperforms another model 80\% of the time, the weights will be 0.8 for the first model and 0.2 for the second model.
This is an indication of model heterogeneity, which can be used as a guide to improve the model.
Thus, stacking is not limited to combine predictions from diferent models.

However, care must be taken when comparing a large number of models to avoid the risk of overfitting.

\section{Initial models}
\label{ch:initial}
This section compares two different categories of initial models, which corresponds to the first step of the workflow in \cite{gelman_bayesian_2020}.
The first one is an extension of model \ref{eq:mod_hb} from \cite{morelli_hierarchical_2021} that does full Bayesian inference on the shift parameter of the log-shift transform.
The second category includes a series of skewed likelihoods, which are tested against the simulation scenarios from section \ref{ch:simulations} through posterior predictive tests.
At the end of this section, there is a short discussion on the adequacy of each type model when dealing with unimodal, skewed, leptokurtic data.

\section{Data transformation of the dependent variable}

At the unit level, income is usually a right-skewed distribution.
As measures of poverty and inequality are usually based on income, this characteristic shape poses challenges to model specification.
A classical linear regression with Gaussian errors will not be able to capture key aspects of the dependent variable due to its skewness.
A possible solution would be to use a regression with a skewed likelihood like a gamma, lognormal or skew-normal distribution.
In practice, those distributions have strong limitations: the logarithm of the lognormal has to be a normal distribution and the skewness of the skew-normal distribution is limited to the range $(-1, 1)$.
The inadequacy of these distribution to model income is shown in appendix ??? through posterior predictive checks.

\subsection{Log-shift transformation and skewness}
A common transformation for income in economic applications is the natural logarithm of the form $\log(y + \lambda)$. However, there might still be some skewness left, which is exacerbated by very low incomes.
If there are a lot of units with incomes between zero, there will likely be a long tail to the in the left side of the transformed distribution.
As \cite{rojas_perilla_data_2020} point out, it is possible to add a fixed term $s$ inside the logarithm so that $y+s \ge 1$ to avoid problems when $0 \le y \le 1$,
but the transformed variable might still be highly skewed.
Moreover, log-income is usually heavy-tailed.
This is not surprising, as a variable has to be distributed according to a log-normal distribution for its logarithm to be normally distributed.
As already mentioned, the log-normal distribution is usually not a good fit for income data.

To bring the dependent variable closer to a normal distribution, \cite{rojas_perilla_data_2020} explore different types of data-driven transformations such as Box-Cox or Yeo-Johnson.
While effective, these transformations are piecewise functions, which adds an additional layer of complexity when backtransforming to the original scale.
Another more simple data-driven transformation described by \cite{rojas_perilla_data_2020} is the log-shift defined as $y^* = \log(y + \lambda)$, where $y$ is the original variable and $\lambda$ is the shift term. Although $s$ and $\lambda$ fulfill similar purposes, they have different meanings: $s$ is a fixed term chosen in advance, whereas $\lambda$ is a parameter to be estimated.
By adjusting $\lambda$, it is possible to make the transformed variable more symmetric.
Moreover, the backtransformation is straighforward: $y = e^{y^*} - \lambda$.
However, the transformed variable might still have considerable excess kurtosis, even though it might still be almost symmetric.
\cite{rojas_perilla_data_2020} point out that minimizing skewness (Royston Lambert 2011) is just one approach to estimating $\lambda$.
One can also aim to minimize the distance (e.g. Kolomogorov-Smirnov or Cramér-von Mises) to another distribution, usually the Gaussian.
Their preferred approach is to maximize the REML of the model under data-transformations.
For the method proposed in the present paper, it is only necessary to minimize skewness to better fit the symmetric likelihood chosen for the model in the transformed scale.
Finally, note that the logarithm is a monotonic transform.
Therefore, this approach only works when the mode of the distribution in not at one extreme of the distribution support.
Specifically, this means that distributions that have a mode at their minimum like the exponential or Pareto will not become more symmetric after taking the logarithm and adding a shift term.

There is a further question related to dependent variable skewness in the linear mixed model.
\cite{rojas_perilla_data_2020} propose to a pooled skewness measure that weighs the skewness of the unit and area-level error according to their variances.
In principle, it is possible that skewness not only affects the unit-level error $\varepsilon_{di}$, but also the area-level error $u_{d}$.
While right-skewness is a common pattern of income at the unit-level (a few individuals/households earn much more than the rest), the picture is less clear at the area-level, as the areas can be defined in very different ways.
For example, if areas are defined as municipalities, there is a certain degree of arbitrariness to geographic boundaries:
although the Mexican states of Guerrero and Baja California have roughly the same area, the former has 81 municipalities while the latter only has six.
Any distribution of $u_d$ will reflect primarily the arbitrary subdivision rather than an underlying economic phenomenon.
Therefore, only the skewness at the unit-level errors is considered in this paper.
The Bayesian model proposed assumes that $u_d$ follows a normal distribution and is therefore symmetric by definition.

\subsection{Estimating the shift parameter}
A key question is how to estimate the shift term $\lambda$.
There are two options: estimate it from the data in an empirical Bayes way or do full Bayesian inference.
Both approaches have their advantages and disadvantages.
When estimating $\lambda$ from the data, the aim is to reduce skewness as much as possible.
This can be done by minimizing the absolute empirical skewness of $y^*$, or at least bringing it below a predetermined threshold.
This approach has the advantage that it is straightforward to control the skewness of $y^*$.
However, the uncertainty of estimating the shift parameter is not taken into account by not including the shift parameter into the model.

Integrating the shift parameter naively into the model might lead to estimation problems.
Defining the prior directly on $\lambda$ is not straightforward, as each different distributions might lead to  and with no further prior constraints the Markov chains might get stuck and not mix well.
As discussed in the previous section, minimizing the skewness of the transformed variable close to zero is likely to improve the performance of a symmetric likelihood distribution.
In the Bayesian context, this can be done by including a very tight prior around zero over skewness
\begin{equation*}
    S \sim \mathcal N(0, \delta).
\end{equation*}
Here, $\delta > 0$ is a small positive constant and $S$ is skewness defined as
\begin{equation*}
    \displaystyle S =  \frac{\frac 1 N \sum^{N}_{i = 0} (y_i^* - \bar y^* )^3}
    {\left[ \frac{1}{N - 1} \sum^{N}_{i = 0} (y_i^* - \bar y^* )^2 \right]^{3/2}},
\end{equation*}
where $y^* = \log(y + \lambda)$ is the transformed dependent variable. Thus, the prior on $S$ indirectly defines a prior on $\lambda$.
While it is not necessary to formulate a prior directly $\lambda$, it still recommended to define the lower bound of $\lambda$ as $-\min(y)$ in the programming framework as a safety check, as to avoid negative values inside the logarithm function.
In practice, the Markov chain steers clear of regions too close to the minimum for $\lambda$ after warmup iterations.

The hyperparameter $\delta$ controls the deviation from the zero skewness constraint.
Values around $10^{-4}$ worked well in simulation experiments and the posterior values for $S$ are usually very close to zero, but far beyond the implied 95\% prior region of $S \pm 2 10^{-4}$.
Nevertheless, there are two potential problems to be aware of.
If $\delta$ is too small, then skewness might be reduced too much.
This would be equivalent to overfitting the skewness of the training data, which by no means reflects the skewness for out-of-sample data.
However, if $\delta$ is too large, then many unrealistic values for $\lambda$ will be allowed, which might lead to poorly mixing Markov chains with a high R-hat (> 1.05).
By plotting the posterior density of $S$ and using Bayesian diagnostic tools such as R-hat or posterior predictive checks, it is possible to assess whether $\delta$ is too small or too large.

(EXAMPLES PLOTS for too large/small $\delta$?)

\subsection{Adjusting the Jacobian of the likelihood}

The transform of the dependent variable introduces a distortion that makes it necessary to adjust the Jacobian of the likelihood using Jacobi's transformation formula (12.6, Jacod Protter / Theorem 1.101 Klenke).
Let $X$ be a random variable defined over $I \subseteq \mathbb{R}$ with density $f^X$. For a continuous differentiable function $\varphi: I \rightarrow \mathbb{R}$ and $\varphi' \ne 0$ for all $x \in I$, define $Y := \varphi(X)$.
The density of the random variable $Y$ can then be defined as
\begin{equation*}
    f^Y(y) = f^X(\varphi^{-1}(y))
    \left|\displaystyle \frac{d \varphi^{-1}(y)}{dy}\right|
    \mathbb{I}(y \in \varphi(I)), ~~ y \in \mathbb{R},
\end{equation*}
where $\mathbb{I}(\cdot)$ is the indicator function.
Assuming that $y$ represents the dependent variable in the original scale, then $\varphi^{-1}(y) = \log(y + \lambda)$ and the Jacobian adjustment $\displaystyle \frac{d \varphi^{-1}(y)}{dy} = \frac{1}{y + \lambda}$, which is positive due to $y + \lambda > 0$. As the transformation parameter $\lambda$ is chosen to drastically reduce the skewness of $\log(y + \lambda)$, the density $f^X$ is chosen to be from a symmetric distribution – e.g., a Student's $t$-distribution. The likelihood of $y$ under the log-shift transformation is therefore
\begin{equation}
    \displaystyle f^Y(y) = \text{Student}(\log(y + \lambda)| \mu , \sigma, \nu)
    \cdot \frac{1}{y + \lambda}
    \cdot\mathbb{I}(y \ge \lambda), ~~ y \in \mathbb{R}.
\end{equation}

\subsection{Redifining the linear mixed model}

The Jacobian adjustment in the likelihood due to the log-shift transformation leads to a reformulation of model XY as follows:



\subsection{Alternative 2: Skewed likelihoods}
\label{ch:skewed_likelihoods}
A natural question raised by income data is whether a skewed likelihood might provide the best results.
In line with the Bayesian workflow proposed by \cite{gelman_bayesian_2020}, this paper is not limited to a single initial model based on data-driven transformations, but also explores the impact of using skewed likelihoods.
The following skewed likelihoods were taken into account: gamma with logarithmic link, gamma with
softplus link, lognormal, skew-normal and exponentially modified Gaussian (exGaussian).
This variety goes beyond the distribution shape, but it also affects the assumed relation between dependent variable and predictors.
The gamma with lo link and lognormal distributions imply a multiplicative model, whereas the rest assume that the predictors have an additive impact on the dependent variable.

The results of these alternative models are presented in this section.
The package \code{brms} \citep{burkner_brms_2017}, which provides a more user-friendly interface to \code{Stan}, is used to do a first check of the likelihoods.
This follows a principle from \cite{gelman_bayesian_2020}: when working with preliminary models, it is better to use tools that allow for quick checks.
\code{brms} already has a number of likelihoods implemented, and if modifications are needed, it is always possible to take the \code{Stan} code and modify it.
But starting coding from scratch with \code{Stan} would take longer, especially because working with different likelihoods, requires different parametrizations.

% PPC: skew-normal
\begin{figure}[t]
    \centering
    \begin{subfigure}{0.29\textwidth}
        \includegraphics[width=\textwidth]{./graphics/skewed_likelihood/skew_normal/logscale_smp}
        \caption{lognormal}
    \end{subfigure}
    \begin{subfigure}{0.29\textwidth}
        \includegraphics[width=\textwidth]{./graphics/skewed_likelihood/skew_normal/gb2_smp}
        \caption{GB2}
    \end{subfigure}
    \begin{subfigure}{0.29\textwidth}
        \includegraphics[width=\textwidth]{./graphics/skewed_likelihood/skew_normal/pareto_smp}
        \caption{Pareto}
    \end{subfigure}

    \caption[Posterior predictive check for skew-normal likelihood]{Posterior predictive check for skew-normal likelihood. The black line is the density plot of the dependent variable for the respective simulation scenario. The blue density plots represent 100 MCMC draws from the posterior predictive distribution.}
    \label{fig:skewnormal_ppc}
\end{figure}

Some of this likelihoods proved to be very poor fits.
The skew-normal likelihood (Figure \ref{fig:skewnormal_ppc}) captures well the main characteristics of the GB2 scenario. On the other hand, it is not adequate for the logscale scenario, which is characterized by a much higher skewness.
This is due to the limitations of the skew-normal distribution, as it has a maximum skewness of $\pm 1$.
Moreover, in this scenario the model produces some negative predictions.
In the Pareto scenario, the model only converged for the version with some out-of-sample areas.
This indicates that the model might need to be reparametrized and the priors adjusted for it to work properly.
However, because this likelihood was already shown to be unreliable for the log-scale scenario, the model with a skew-normal likelihood is not further developed.

% PPC: exGaussian
\begin{figure}[h]
    \centering
    \begin{subfigure}{0.29\textwidth}
        \includegraphics[width=\textwidth]{./graphics/skewed_likelihood/exgaussian/logscale_smp}
        \caption{lognormal}
    \end{subfigure}
    \begin{subfigure}{0.29\textwidth}
        \includegraphics[width=\textwidth]{./graphics/skewed_likelihood/exgaussian/gb2_smp}
        \caption{GB2}
    \end{subfigure}
    \begin{subfigure}{0.29\textwidth}
        \includegraphics[width=\textwidth]{./graphics/skewed_likelihood/exgaussian/pareto_smp}
        \caption{Pareto}
    \end{subfigure}

    \caption[Posterior predictive check for exGaussian likelihood]{Posterior predictive check for exGaussian likelihood. The black line is the density plot of the dependent variable for the respective simulation scenario. The blue density plots represent 100 MCMC draws from the posterior predictive distribution. Note that the Markov chains have not mixed well in the lognormal and GB2 scenarios.}
    \label{fig:exgaussian_ppc}
\end{figure}

\begin{figure}[t]
    \centering
    \begin{subfigure}{0.29\textwidth}
        \includegraphics[width=\textwidth]{./graphics/skewed_likelihood/gamma_log/logscale_smp}
        \caption{lognormal}
    \end{subfigure}
    \begin{subfigure}{0.29\textwidth}
        \includegraphics[width=\textwidth]{./graphics/skewed_likelihood/gamma_log/gb2_smp}
        \caption{GB2}
    \end{subfigure}
    \begin{subfigure}{0.29\textwidth}
        \includegraphics[width=\textwidth]{./graphics/skewed_likelihood/gamma_log/pareto_smp}
        \caption{Pareto}
    \end{subfigure}

    \caption[Posterior predictive check for gamma likelihood]{Posterior predictive check for gamma likelihood with logarithmic link. The black line is the density plot of the dependent variable for the respective simulation scenario. The blue density plots represent 100 MCMC draws from the posterior predictive distribution.}
    \label{fig:gamma_ppc}
\end{figure}

\begin{figure}[h]
    \centering
    \begin{subfigure}{0.29\textwidth}
        \includegraphics[width=\textwidth]{./graphics/skewed_likelihood/lognormal/logscale_smp}
        \caption{lognormal}
    \end{subfigure}
    \begin{subfigure}{0.29\textwidth}
        \includegraphics[width=\textwidth]{./graphics/skewed_likelihood/lognormal/gb2_smp}
        \caption{GB2}
    \end{subfigure}
    \begin{subfigure}{0.29\textwidth}
        \includegraphics[width=\textwidth]{./graphics/skewed_likelihood/lognormal/pareto_smp}
        \caption{Pareto}
    \end{subfigure}

    \caption[Posterior predictive check for lognormal likelihood]{Posterior predictive check for lognormal likelihood. The black line is the density plot of the dependent variable for the respective simulation scenario. The blue density plots represent 100 MCMC draws from the posterior predictive distribution.}
    \label{fig:lognormal_ppc}
\end{figure}

The exGaussian scenario (Figure \ref{fig:exgaussian_ppc}) produces negative predictions in some scenarios and it is not able to capture the main features of the distributions.
While the exGaussian does a good job in the Pareto scenario, it fails on the other scenarios.
In figure Xa and Xb, it can be seen that the algorithm does not converge, with two different posterior predictive as a result.
Placing a more restrictive prior on the coefficients did not alleviate the convergence problem.
However, none of these two distributions adequately captures the main features of the dependent variable in the logscale and GB2 scenarios.
Therefore, this version of the likelihood was also discarded.


The gamma likelihood with a logarithmic link (Figure \ref{fig:lognormal_ppc}) performs quite well on all three scenarios.
Only in the GB2 scenario the posterior predictive draws are flatter than the dependent variable.
Despite its good performance, this model has two drawbacks when fitted to the survey data from Mexico.
First, it takes longer to fit than the log-shifted model presented in this paper, which is likely to be a consequence of the long tails of the distribution.
Second, it is highly dependent on the lowest values.
While in a real income survey there are observations quite close to zero, in the simulation scenarios the minimimum values tend to be somewhat higher.
Due to the non-linear exponential transformation it becomes highly dependent on where exactly the distribution is.
A gamma likelihood with a softplus link function (not shown) produced extremely small predictions, far away from the original order of magnitude and was therefore not further taken into consideration.
Note that the gamma likelihood with a softplus link, as well as the skew-normal and exGaussian distribution, allow for an additive model.
In contrast, distributions that use a log link assume a multiplicative model.

An additional check is provided by a lognormal likelihood (Figure \ref{fig:lognormal_ppc}), which performs well in all three scenarios according to the posterior predictive check.
Nevertheless, the logarithm of a lognormal variable should be normally distributed, which is rarely the case for income data.
Therefore, it is not deemed to be a viable alternative.
\begin{figure}[t]
    \centering
    \begin{subfigure}{0.32\textwidth}
        \includegraphics[width=\textwidth]{./graphics/skewed_likelihood/gamma_log/2d_logscale_smp}
        \caption{logscale}
    \end{subfigure}
    \begin{subfigure}{0.32\textwidth}
        \includegraphics[width=\textwidth]{./graphics/skewed_likelihood/gamma_log/2d_gb2_smp}
        \caption{GB2}
    \end{subfigure}
    \begin{subfigure}{0.32\textwidth}
        \includegraphics[width=\textwidth]{./graphics/skewed_likelihood/gamma_log/2d_pareto_smp}
        \caption{Pareto}
    \end{subfigure}
    \label{fig:gamma_2d}
    \caption[Posterior predictive check with test statistics for gamma likelihood]{Posterior predictive check with test statistics for gamma likelihood with logarithmic link for the logscale, GB2 and Pareto scenarios. The top row shows the meand and the standard deviation, the middle row the median and the IQR and the bottom row the 10\% and 90\% quantiles. The black dot represents the value for the dependent variable, while each one of the blue dots represents the respective summary statistic calculated for each one of the 2000 draws from the Markov chain.}
\end{figure}

As the gamma likelihood with a logarithmic link provided the best fit among the skewed distributions, additional posterior predictive checks are shown in Figure \ref{fig:gamma_2d}.
The two main parameters fitted in a gamma likelihood are the shape and the scale, which directly impact the expected value an the variance.
Thus, it is not suprising that this model performed well in capturing the mean and the standard deviation (top row).
Note the correlation between mean and standard deviation, due to the fact that the coefficient of variation ($\mu / \sigma^2$) is always equal to the shape parameter of the gamma distribution.
The second and third rows of Figure \ref{fig:gamma_2d} uncover more details on the fitting process, because summary statistics such as quantiles are ancillary do not depend directly on the distribution parameters.
In the logscale scenario, which is multiplicative, the model does a good job in capturing the median, the IQR, as well as the 10\% and 90\% quantiles.
In the other two scenarios, the data median is represented well only in the GB2 scenario.
In the Pareto case, the median from the data is somewhat higher than in most of the Monte Carlo simulations.
The estimates for the other test statistics (IQR, 10\% and 90\% quantiles) diverge between the data and the simulations from the posterior predictive distribution.
The 90\% quantile in the dependent variable is lower than in the simulated estimates.
On the other hand, the 10\% quantile is higher than most Monte Carlo simulations.
This shows the model has some trouble with regions further away from the median, which is especially clear in the IQR that is larger in the simulations than in the dependent variables.
The fat right tail in the simulations can also be seen in Figure \ref{fig:gamma_ppc}.
When estimating FGT estimators, it is less relevant whether quantiles above the median can be approximated well, as all observations above 60\% of the median are ignored.
Therefore, it is most important to estimate the median accurately and this is provided by the gamma likelihood with log link.

There are other skewed distributions, where the mode is equal to the minimum.
Some examples include the exponential, the Pareto the chi-squared and the half-normal distributions.
Both the exponential and the chi-squared are special cases of the gamma distribution, which makes them less flexible as likelihood than using a plain gamma distribution.
The half normal has very thin tails and a very low skewness that does not correspond with the long tails observed empirically in income data.

While the main model used in this paper uses a different approach, this does not mean that the use of skewed likelihoods is wrong.
The most promising likelihood was the gamma distribution with a log link.
With this likelihood, it would be possible to avoid very extreme predictions when backtransforming the simulations from the posterior predictive distribution.
However, the heavy right tail might lead to a long model estimation time.
The skew-normal likelihood does not seem to be able to capture the skewness of typical income data, due to its limited maximum skewness.
The exGaussian distribution might be more adequate in this context, but this question is beyond the scope of this paper.



\subsection{Initial model comparison}

In principle, it is possible to keep developing the two alternative models in parallel and then compare them at the end of the workflow, but this implies exploring additional dimension of complexity in the model space.
Therefore, one of both alternatives should be chosen based on the flexibility and limitations of each approach.

The log-shift scenario has two main disadvantages.
By using a log-shift transformation, the user does not exactly now the analytical form of the density in the backtransformed scale.
Moreover, the medians of the posterior predicitve samples in the Pareto scenario are relatively close to the data, but still systematically lower.
Another problem is represented by the very extreme predictions that can arise due to backtransforming samples from the heavy-tailed Student's $t$-distribution.

With respect to the skewed likelihoods, two similar distributions performed well in all three scenarios: the lognormal and the gamma with a log link.
However, these two distribution have theoretical limitations.
On the one hand, the logarithm of a lognormal variable has to be be normally distributed, which is rarely the case for income data.
For this reason, it is not deemed to be a realistic alternative.
On the other hand, a gamma distribution has a constant coefficient of variation, which depends on the shape parameter.
Additionally, the logarithm of a gamma distribution is either left-skewed or symmetric, which might not necessarily be realistic for income data.

Whereas the gamma and lognormal distributions only have two parameters (shape/rate or mean/variance, respectively), the main advantage of the log-shift model \ref{eq:trafo_hb} is that it offers more flexibility due to the additional parameters.
The Student's $t$-distribution is not only parametrized in terms of mean and scale, but also of degrees of freedom.
Besides, the log-shift distribution adds another parameter $\lambda$ to the likelihood through the transformed dependent variable.
This flexibility does not come at a high cost of interpretability that characterizes more complex distributions.
While the less than ideal performance in the Pareto scenario is problematic, there are two factors which should be considered.
First, although the prior parametrization was taken mostly from \cite{morelli_hierarchical_2021},
this can be improved through the use of prior predictive checks (see chapter \ref{ch:coef_var_spec}).
Second, in this scenario the degrees of freedom were below two, which is not ideal, due to the fact that the Student's $t$-distribution has an infinite variance for $\nu \le 2$ and an undefined variance for $\nu \le 1$.
This extreme behavior might cause problems in the Pareto scenario.
There are two simple solutions to this problem.
Either $\nu$ is set to a constant (e.g., a value between 2 and 3) when the posterior of $\nu$ is clearly below 2, or the model is reparametrized so that $\nu$ does not take values below zero.


In light of this discussion, the log-shift model \ref{eq:trafo_hb} is chosen for the rest of the present paper, as it provides a good balance between interpretability, ease of parametrization and flexibility.
Additionally, it extends the literature on data-driven transformation in the line of \cite{rojas_perilla_data_2020} to the Bayesian paradigm.
This does not mean that the use of skewed likelihoods is an inferior approach, only that its use with income data is left for future research.
A feature of the gamma likelihood that can be useful is that its mean can be parametrized in the original scale even when using a log link as done in the \code{brms} package.
This is particularly relevant, if benchmarking \citep{pfeffermann_new_2013} should be included as a module in the Bayesian model.
For example, a somewhat narrow prior can be placed on the mean of all domain estimates so that it roughly matches the benchmarking mean based on the direct estimates.
The integration of benchmarking into the model is not further considered in this paper.


\section{Specification of coefficient and variance priors}
\label{ch:coef_var_spec}

After choosing the Student's $t$-likelihood with a log-shift transformation, the next step reevaluates the plausibility of the priors introduced in model \ref{eq:trafo_hb} – the extension from the model in \cite{morelli_hierarchical_2021}.
This section presents the first iterative improvement of the model.
For this modification of the model (step 6 in the workflow), prior predictive checks are done with the simulation scenarios to avoid using the data multiple times.

Appendix \ref{appendix:coeff} shows that even with a shift term in the logarithmic transformation, the coefficients can be interpreted as the approximate percentage change of the dependent variable in the original scale.
This approximation holds mostly for coefficients between -0.3 and 0.3.
However, a change of around 30\% in the original scale for each additional unit is itself very high and this observation can be included in the prior distribution.
Moreover, note that through the use of the logarithmic transform the covariates have a multiplicative effect on the original scale.
Thus, somewhat large coefficients in the logarithmic scale can have a huge impact when backtransforming to the original scale.
For the independent priors, two different types of distribution can be considered, either a normal distribution or a heavy-tailed distribution such as the Student's $t$ with 3 degrees of freedom.
In this case, a prior distribution with heavy tails is not as desirable, because even if most of the probability mass is contained in the interval between -0.3 and 0.3, the prior would allow more extreme values than a thinner-tailed distribution.
This can be avoided by choosing a Gaussian distribution, which does not place as much probability mass on extreme values.
The coefficient prior is parametrized to have a mean of zero and a standard deviation of 0.2, which implies that the 5\% and 95\% quantiles are around -0.3 and 0.3 respectively.
To ensure that the samples from the prior predictive distribution are in a realistic range, the prior for the intercept is set to a relatively $\mathcal N (4, 3)$.
In the final model, the intercept is fit to the data and the prior can be less informative, e.g., $\mathcal N(0, 5)$.

Another key element of the models is how the priors are defined for the standard deviations at the unit-level ($\sigma_e$) and area-level ($\sigma_u$), as these have a large impact on the implied dispersion in the dependent variable.
As the likelihood is a generalized Student's $t$-distribution, the parameter $\sigma_e$ cannot be interpreted directly as the standard deviation of the distribution.
Given a random variable $Y$ that follows this distribution, the variance is given by $Var(Y) = \sigma_e^2 \frac{\nu}{\nu - 2}$.
To reason more easily about the unit-level variance a new parameter $\sigma = \sqrt{Var(Y)}$ is introduced so that $\sigma_e = \sigma \sqrt{\frac{\nu - 2}{\nu}}$.
The effect of different values for the rate parameter of the gamma prior on $\sigma$ and $\sigma_e$ is explored in the prior predictive checks.

\begin{figure}

    \begin{subfigure}{\linewidth}
        \centering
        \includegraphics[width=0.72\textwidth]{./graphics/prior_predictive_checks/prior_check_gb2_start}
        \caption{Scale prior: $Ga(2, 0.75)$}
        \label{fig:ppc_start}
    \end{subfigure}


    \begin{subfigure}{\linewidth}
        \centering
        \includegraphics[width=0.72\textwidth]{./graphics/prior_predictive_checks/prior_check_gb2_tight}
        \caption{Scale prior: $Ga(2, 7)$}
        \label{fig:ppc_tight}
    \end{subfigure}
    \caption[Prior predictive checks for scale parameters in the GB2 scenario.]{Prior predictive checks for scale parameters $\sigma$ and $\sigma_u$ in the GB2 scenario. In the scatterplots, the logarithm dependent variable is plotted against the corresponding sample from the prior predictive distribution. Note the different scaling of the x-axes.}
    \label{fig:prior_pred_variance}
\end{figure}

For the prior predictive checks, the shift term is dropped.
The shift term depends on the dependent variable so it is not possible to sample from its prior distribution.
Besides, it amounts to an additive constant in the backtransformed scale ($e^{y^*} - \lambda$). The impact of the exponential function is therefore much larger than the effect of the shift term $\lambda$.
The range of Mexican income data is in the tens of thousands, which is also captured in the simulation scenarios.
Therefore, the prior predictive simulations should not be much higher than 12 in the logarithmic scale (which corresponds to around 160000 pesos in the original scale).
Conversely, as the real data has almost no observations below 1, the values in the logarithmic scale should not go far below zero.
Finally, note that the degrees of freedom in the Student's $t$-distribution can lead to very extreme simulations when $\nu$ is low.
As a consequence, the prior for $\nu$ is changed to $Ga(2, 2)$ \textit{only} for the prior predictive check to focus on the impact of heavier tails.
The model is reparametrized so that the minimum value for $\nu$ is 2. This ensures that the variance of the distribution is still finite.
The assumption here is that if the prior predictive simulations are not too extreme for low degrees of freedom, they will also be in a reasonable range for high values of $\nu$, for which the likelihood has thinner tails.


The results for the prior predictive check in the GB2 scenario can be seen in Figure \ref{fig:prior_pred_variance}.
Other scenarios produced very similar results.
The prior predictive checks are shown in form of scatterplots, where the dependent variable is plotted against samples from the prior predictive distributions.
These samples do not have to fit perfectly the dependent variable, but they should be in a realistic range.
The data is shown in the logarithmic scale.
The aim is to determine which value of the rate parameter is best for both scale parameters $\sigma$ and $\sigma_u$.
Figure \ref{fig:ppc_start} shows the scatterplots for a rate parameter value equal to 0.75, while Figure \ref{fig:ppc_tight} displays the results for a tighter rate value of 7.
Note the different scaling of the x-axes.
For a rate parameter of 0.75, there are samples that are extremely high, above 50 in the logarithmic scale.
On the other hand, a rate parameter of 7, which implies a much tighter prior is much more realistic – especially, because only very few samples are much higher than 10 and even samples with higher values are not as high as in the first specification.
Prior predictive checks with a very wide scale prior can be found in appendix \ref{appendix:wide_prior}.

After the prior predictive check, it is possible to conclude that the tighter scale prior provides the most realistic results, while still allowing for a few higher than expected but not extreme predictions.
Model \ref{eq:trafo_hb} can thus be reformulated as:
\begin{equation}
    \begin{split}
        p(\log(y_{di} + \lambda) |\boldsymbol \beta, u_d, \sigma_e, \nu)   =        \text{Student}&(\log(y_{di} + \lambda)| \boldsymbol{x'}_{di} \boldsymbol \beta + u_d,\ \sigma_e\ , \nu)\cdot (y_{di} + \lambda)^{-1}, \\
        u_d | \sigma_u & \sim \mathcal N(0, \sigma_u),\ d = 1, ..., D, \\
        \beta_0 & \sim \mathcal N (0, 5),\\
        \beta_k & \sim \mathcal N(0, 0.2),\ k = 1, ..., K,\\
        \tilde \nu & \sim Ga(2, 0.1), \\
        \nu & = \tilde \nu + 2,\\
        \sigma_u & \sim Ga(2, 7), \\
        \sigma & \sim Ga(2, 7), \\
        \sigma_e & = \sigma \sqrt{\frac{\nu - 2}{\nu}},\\
        S(\log(y_{di} + \lambda)) & \sim \mathcal N(0, 0.01),\\
    \end{split}
    \label{eq:trafo_coef_var}
\end{equation}
where $\ d = 1, ..., D,\ i = 1, ..., N_d$. The parameter $\sigma$ is now explicitly included into the model and a new parameter $\tilde \nu$ is introduced to enforce that $\nu$ does not take values below 2. The intercept is now explicitly included into the model as $\beta_0$.







\section{Variable selection with the horseshoe prior}

A key step in specifying the model is deciding which variables to use.
However, this is at the same time one of the more challenging steps.
A simple such as forward or backward regression might be problematic (why? reference).
The ideal solution would be to estimate models with all possible variable combinations and then compare the predictive quality of the models – e.g., through cross-validaton.
However, this solution is not feasible computationally due to the very large number of possible models.
Another approach is to use shrinkage methods to check which variables can be left out without significantly compromising the predictive power of the model.
If a coefficient is very close to zero, it is a sign that the respective variable might be removed.
In this section, the regularized horseshoe prior \citep{piironen_sparsity_2017} is presented and then used to select relevant variables. The selection is done by comparing the PSIS-LOO of models with the less impactful (CHANGE WORD) variables removed.

\subsection{Horseshoe prior: theory}

In the Bayesian framework, shrinkage can be best understood in the context of the prior.
The main idea is to have a narrow region of high density around zero that shrink coefficients,
while at the same time including fat tails to allow some coefficients to deviate from zero over a wider range.
In its most extreme formulation, it corresponds to the spike-and-slab prior introduced originally by (CITATION).
While this prior has a hight theoretical relevance, it leads to computational problems (WHICH?).
The horseshoe prior, originally introduced by \cite{carvalho_horseshoe_2010}, aims to solve these problems.
The horseshoe prior can be formulated as follows:

In practice, the original horseshoe prior leads to computational problems in the form of HMC divergences.
 \cite{piironen_sparsity_2017} address this problem by introducing a regularized horseshoe prior.


\section{Specification of random effect}
\label{ch:raneff}
In applications of small area estimation methods, the random effect is usually defined as the domain for which the indicators should be calculated.
For example, \cite{rojas_perilla_data_2020} use a random effect at the municipality level in the model, as the poverty indicators of interest have to be calculated for each municipality.
However, this approach poses two problems.
First, the information contained in the survey design might not be used.
There is no guarantee that the random effect is meaningful from the perspective of how the sample was constructed.
Second, it might lead to very sparse areas with lots of out-of-sample areas.
In this section, the traditional way of defining the random effect is compared to an approach that approximates the stratification in the survey design.

Section XY included a succint introduction to the survey design of the data from the Mexican state of Guerrero.
On closer inspection, municipalities play no role whatsoever in the sampling process of the households.
Instead, the population is stratified by federal state, by geographic region (urban or rural) and also by socio-economic indicators taken from the previous census.
Changing the definition of the random intercept does not have an adverse impact on the calculation of FGT indicators.
As the data is a the unit-level and each observation has information on which municipality it belongs to, it is not necessary that the random effect coincides with the municipality.
Predictions for income at the unit-level can be generated from any model and the FGT indicator can be calculated simply by using the municipality as a grouping variable.

A major challenge is to find a structure that can be found both in the survey and the census.
While the survey includes a variable that indicates the stratum to which a given observation belongs, this cannot be matched with the stratum in the census.
Therefore, the stratification variable in the data cannot be used directly.
However, there are some variables in both the survey and the census that can be used to approximate the stratification procedure.
Besides the \code{rururb} variable that indicates whether the observation corresponds to a rural or urban area, there are four additional binary variables (Table \ref{tab:disadvantages}), which contain information of disadvantage in areas such as education, health care, housing quality and access to public utilities.
As all five variables are binary, there are $2^5 = 32$ possible combinations.
Each one of these 32 combinations is now considered as a domain that is used to define the varying intercept in the model.
How sparse? Do we still have out-of-sample areas?

Using the priors defined in the previous section, the two definitions of the

How does it fare in terms of predictive power (PSIS-LOO)?

The results from PSIS-LOO tell us that the models are almost indistinguishable and that the alternative definition of the random effect might perform slightly better.
Even if the models are not substantially different in predictive terms, there is one substantial advantage of using the alternative specification: there are no out-of-sample domains.
This reduces drastically the uncertainty in the predictions for out-of-sample municipalities.

Note that there are some limitations to this approach.
First, the stratified structure is only an approximation.
Second, there are still other dimensions that are not taken into account that might still be problematic, e.g., the clustering that comes with using primary sampling units (besser erklären).
Third, this is an example that is limited to the data from Mexico.
Nevertheless, the main insight from this section is that for unit-level data the random effect does not have to follow the level at which indicators are estimated.
Looking for alternatives specifications for the random effect might lead to better predicitive power and to less out-of-sample domains.
\section{Modelling correlations at the area-level}
\label{ch:area_corr}

Until now, the assumption has been that all random effects at the area-level are independent from each other, i.e., $u_d|\sigma_u \sim \mathcal N (0, \sigma_u)$.
However, this assumption seems unplausible.
For example, if the areas are defined as municipalities, it is likely that neighbouring regions have similar characteristics and are therefore correlated.
If on the other hand the areas are the strata considered in the previous section, the 16 strata that correspond to a rural area have more in common with each other than with urban strata.
In this section, a new prior that captures correlations at the area level is introduced to the model.
The result is then compared to the model that was defined in the previous section.
Thus, the domains are not defined as the municipalities, but as the strata presented in the last section.

There are multiple options to model correlation between areas.
The simpler LKJ prior based on the work by \cite{lewandowski_generating_2009} places a restriction over permissible correlation matrices.
Autoregressive approaches such as IAR, CAR and SAR \citep{chung_bayesian_2020} represent another way of capturing spatial correlation and takes into account how similar or close the different domains are.
Moreover, other priors such as the random walk prior used used in \cite{gao_improving_2021} offers an alternative to modelling dependencies between areas.

The resulting model will be compared with model \ref{eq:log_scenario} that assumes independence between areas, but with the modified random effect.
A comparison between all possible priors that capture area correlation is beyond the scope of this paper and left for future research.

\subsection{LKJ prior}

Let $\Sigma$ be a $D \times D$ covariance matrix, so that $u \sim \mathcal{N}(\boldsymbol{0}_D, \Sigma)$, where $\boldsymbol 0_D$ and $u$ are $D$-dimensional vectors.
The density of $u$ can be written as:
\begin{gather*}
    p(u|\Sigma) = (2\pi)^{-\frac D 2}\det(\Sigma)^{-\frac 1 2} e^{(-\frac 1 2 u'\Sigma^{-1} u)}.
\end{gather*}
The covariance matrix $\Sigma$ can be decomposed as $\Sigma = \text{diag}(\tau)\Omega\text{diag}(\tau)$, where $\tau$ is a $D$-dimensional vector of scale factor, $\text{diag}(\tau)$ is a $D \times D$ matrix with $\tau$ as its main diagonal and $\Omega$ is a $D \times D$ correlation matrix.
Up to this point, the assumption has been that there is only one scale parameter $\sigma_u$ for all area-level effects $u_d$ and this assumption is still kept.
Because $\tau_d = \sqrt{\Sigma_{d, d}}, d = 1, ..., D$, the decomposition is simplified to $\Sigma = \sigma_u^2 \Omega$.

The standard deviation for the random effect $\sigma_u^2$, already has the prior $Ga(2, 7)$.
However, the correlation between domains is captured by defining a LKJ prior over the correlation matrix $\Omega$\footnote{In practice, it is common to use the Cholesky decomposition of $\Omega$ to avoid numerical issues when estimating the model.
For clarity, the traditional matrix notation is kept in the paper, but note that the decomposition is used in the accompanying code.}:
\begin{gather*}
    \Omega \sim \text{LKJ}(\eta), ~~ \eta \ge 1.
\end{gather*}
The LKJ correlation distribution is defined so that $p(\Omega|\eta) \propto \det(\Omega)^{\eta - 1}$ \citep[Chapter 1.13]{stan_development_team_stan_2021}.
Note that the determinant of $\Omega$ increases as the correlation between components decreases:
an identity matrix has a determinant of 1, while a correlation matrix consisting only of ones (perfect correlation between components) has a determinant of 0.
For $\eta = 1$, the LKJ prior is a uniform distribution over all possible correlation matrices.
However, due to the fact that the function $f(\eta) = k^{\eta-1}$ for a fixed $k \in (0, 1)$ does not converge uniformly towards zero as $\eta \rightarrow \infty$, a higher $\eta$ puts more probability mass on matrices with a higher determinant, i.e., on matrices with a lower correlation between components.
In the model, $\eta$ is set to 5, which strongly favors matrices with determinants closer to 1, i.e., correlation matrices with a moderate correlation between domains.
Figure \ref{fig:heatmap_lkj} shows four posterior draws for $\Omega$ as heatmaps.
There are correlation values in the range from -0.5 to 0.5, which indicates that there is some correlation between areas.
However, the heatmaps do not reveal any clear correlation pattern.
This will be discussed more in detail in section \ref{ch:comparison_lkj_sar}.
In the next step, the SAR prior is presented and included into the model as an alternative to the LKJ prior.

%The results from the model comparison can be seen in in Table XY.
%The model that allows for correlation between areas has a slightly better performance according to PSIS-LOO than the model with no correlation at all.
%However, it is important to remember that elpd$_{\text{LOO}}$ is just an approximation, because PSIS-LOO quantifies predictive power at the unit-level and not at the area-level, whereas the aim of the model is to generate prediction at the area-level.
%It is likely that the difference between the predictive power of both specifications would be clearer when using leave-one-group-out cross-validation.
%Unfortunately, this is still an area of active research and it is therefore beyond the scope of this paper.
%In any case, the slightly better performance of the model with the LKJ priors is taken as a sign that it captures correlations between strata that improve predictions.
%Eta parameter?





\subsection{SAR prior}
The simultaneous autoregression (SAR) prior as described in \cite{chung_bayesian_2020} is defined on a precision matrix $\Pi$.
The starting point is the $D\times D$ proximity matrix $W$, which contains information on how close two domains are.
If $w_d, d = 1, ..., D$ is defined as the sum of row $d$ of matrix $W$ – i.e., $w_d = \displaystyle \sum_{i = 1}^D w_{di}$ –, then the $D \times D$ matrix $L$ is defined as $L = \text{diag}\{w_d\}_{d=1}^D$.
Thus, it is possible to calculate the row-normalized matrix $\widetilde W = L^{-1}W$, in which all rows sum to 1.
The SAR prior in the context of model \ref{eq:trafo_coef_var} is then given by
\begin{gather*}
    \Pi(\rho) = (I_D - \rho \widetilde W)'(I_D - \rho \widetilde W), ~~ \rho \in (-1, 1),\\
    u \sim \mathcal N(\boldsymbol{0}_D, \sigma_u^2 \Pi^{-1}),\\
    \rho \sim \mathcal U(-1, 1),
\end{gather*}
where $\boldsymbol{0}_D$ is a $D$-dimensional zero vector.
Again, note that in this model $\Pi$ is \textit{precision} and not a \textit{correlation} matrix.
Because the single strata used as domains are coded as a 5-digit binary string (see Section \ref{ch:raneff}), the distance matrix $W$ is defined using the Hamming distance.
In short, the Hamming distance takes two strings of the same length and counts how many digits are different, e.g., 00111 and 10100 have a Hamming distance of 3.
This also ensures that the distance of each stratum from itself is 0.
The maximum distance in this case is 5, because each stratum is code with five binary digits.
Note that when $\rho = 0$ the precision matrix $\Pi$ is equal to the identity matrix, which means that there is no correlation between the domains.
As $\rho$ deviates from 0, the effect of the spatial increases.
The uniform prior is chosen for $\rho$, because it is not possible to know how strong the correlation values are before fitting the model to the data.
The limits $\rho \in (-1, 1)$ are needed so that $\Pi$ is positive definite.
Similarly to the LKJ prior, Figure \ref{fig:heatmap_sar} shows four posterior draws for the correlation matrix.
Here, the actual correlation matrix is depicted, \textit{not} the precision matrix $\Pi$.\footnote{$\Pi^{-1}$ is a covariance matrix, which is rescaled with the \code{cov2cor} function in \code{R} to get the correlation matrix. The scaling parameter $\sigma^2_u$ is ignored in the rescaling as it is only a multiplicative constant of the covariance matrix.}
The pattern created by the distance matrix is very clear and will be discussed more in detail in the next section.
Finally, although the prior on $\rho$ is uniform, its posterior distribution (not shown) is right skewed, with 70\% of the probability mass between -1 and 0 and the rest between 0 and 1.
It is not straightforward to interpret the effect of this parameter for large matrices, as the precision matrix $\Pi$ has to be inverted at a later stage.
However, a clear deviation from 0 (as in this case) indicates area-level correlation.

\begin{figure}
    \begin{subfigure}{\linewidth}
        \centering
        \includegraphics[width=0.85\textwidth]{./graphics/rand_intercept/lkj_corr_plot}
        \caption{LKJ(5) prior}
        \label{fig:heatmap_lkj}
    \end{subfigure}
    \begin{subfigure}{\linewidth}
        \centering
        \includegraphics[width=0.85\textwidth]{./graphics/rand_intercept/sar_corr_plot}
        \caption{SAR prior}
        \label{fig:heatmap_sar}
    \end{subfigure}
    \caption[Correlation matrices for the LKJ and SAR priors.]{Four samples of the posterior correlation matrices for the models with LKJ and SAR priors. The domain coding in the axes is explained in section \ref{ch:raneff}.}
    \label{fig:corr_heatmap}
\end{figure}

\subsection{Comparison of LKJ and SAR priors}
\label{ch:comparison_lkj_sar}
The main difference between both correlation priors is that the SAR prior uses actual information on how similar or close the domains are.
In contrast, the LKJ prior is only a prior on the admissible correlation matrices.
The difference becomes clear in Figure \ref{fig:corr_heatmap}, which shows four posterior samples of the correlation matrices from the models with the LKJ and SAR priors.
The axes display the respective domains, whose notation as a 5-digit binary string (e.g., 01100) was explained in section \ref{ch:raneff}.
The correlation matrices produced by the LKJ prior display a completely random pattern with respect to the correlations.
On the othe side, with the SAR prior there is a clear structure in the correlation matrices stemming from the distance matrix $W$.
Even though in Figure \ref{fig:heatmap_map} the correlation values vary from sample to sample, there is a very clear symmetric visible for each sample, which is related to the way Hamming distance used to define $W$.\footnote{Figure \ref{fig:corr_heatmap} depicts only four posterior samples from more than 1000 MCMC samples. The correlation pattern in the SAR prior is present in all samples, but the range of correlation values from just four posterior samples is not representative.}
The nature of the pattern can be further understood in Figure \ref{fig:corr_density}, which shows the distributions of the correlation values for each one of the posterior samples in Figure \ref{fig:corr_heatmap}.
The LKJ prior does not force any particular correlation pattern on the data, but it guarantees that the distribution of the correlation values is roughly equal for each sample.
With an increasing $\eta$, the LKJ prior puts more weight on correlation matrices closer to the identity matrix, which leads to tighter distributions for the correlation values.
On the other hand, there is a much clearer pattern in the densities of the SAR prior specifications.
The spikes in the densities are cause by the Hamming distance, which in the present paper has takes only six values, from 0 to 5.
With continuous distances, the densities would be much smoother.
Moreover, the correlation values densities do not overlap as in the LKJ prior.

\begin{figure}
    \begin{subfigure}{0.49\linewidth}
        \includegraphics[width=0.85\textwidth]{./graphics/rand_intercept/lkj_dens_plot}
        \caption{LKJ(5) prior}
    \end{subfigure}
    \begin{subfigure}{0.49\linewidth}
        \includegraphics[width=0.85\textwidth]{./graphics/rand_intercept/sar_dens_plot}
        \caption{SAR prior}
    \end{subfigure}
    \caption[Correlation density for the LKJ and SAR priors]{Correlation density for the four posterior samples of the LKJ and SAR priors shown in Figure \ref{fig:corr_density}. Each sample are displayed in a different color.}
    \label{fig:corr_density}
\end{figure}


The LKJ and SAR specification are compared with model \ref{eq:trafo_coef_var} without domain-level correlation (base model) using PSIS-LOO.
All models use the stratified random effect specification.
The results are shown in Table \ref{tab:lkj_sar_base}.
The elpd$_{\text{loo}}$ of the SAR model is slightly worse than the base model (-0.07), but the standard error of the difference (second column) is more than three times the estimated elpd$_{\text{loo}}$ difference itself (0.24).
According to PSIS-LOO, the base and SAR models are undistinguishable.
The elpd$_{\text{loo}}$ difference for the LKJ specification (-0.72) is more than three times its standard error (0.22).
Neverteheless, it is important to notice that the elpd$_{\text{loo}}$ (third column) are in an almost identical range and that the standard errors are quite large.
Therefore, when looking at the larger picture, the differences between models seem to be marginal.



The main question at this point is why there is no clearer difference between these three specifications.
There are three possible explanations.
First, as some of the areas are quite small, it is difficult to indentify correlations between areas.
It is possible that with more data the correlation could become more visible.
Second, the PSIS-LOO quantifies leave-one-out cross-validation.
However, leaving out only one observation might not make a large difference for the correlations are at the area-level.
PSIS-LOO might not be the right tool to quantify the difference and alternatives such as leave-one-group-out cross-validation could provide a clearer result.
Third, the predictors might have a good predictive power so that area-level correlation does not have such a large impact.
Finally, this result points to a blind spot in the simulation scenarios, because they do not take area-level correlation into account.
Therefore, it is not possible to know for certain, whether the similar results for all three models in Table \ref{tab:lkj_sar_base} is due to lack of data, to the limitations of PSIS-LOO or simply because the area-level correlation is very small.
Simulated data could have been used to check whether the diagnostic tools can capture a difference in predictive power due to area-level correlations.
Unfortunately, due to this limitations of the simulation scenarios from section \ref{ch:simulations} it is not possible to know the exact reason for the similarity between the three model specifications.
A critical discussion of the simulations scenarios can be found in section \ref{ch:adequacy_simulations}.

% latex table generated in R 4.0.5 by xtable 1.8-4 package
% Fri Aug 20 12:38:58 2021
\begin{table}[ht]
    \centering
    \caption{Comparison of LKJ, SAR and base specification with PSIS-LOO.}
    \begin{tabular}{rrrrrrrrr}
        \hline
        & elpd$_{\text{loo}}$ diff. & S.E. diff. & elpd$_{\text{loo}}$ & S.E. elpd$_{\text{loo}}$  \\
        \hline
        Base & 0.00 & 0.00 & -14799.32 & 54.40  \\
        SAR & -0.07 & 0.24 & -14799.39 & 54.45  \\
        LKJ & -0.72 & 0.22 & -14800.05 & 54.40  \\
        \hline
    \end{tabular}

    \label{tab:lkj_sar_base}
\end{table}


In the current model specification, the random intercept was redifined so that there are no out-of-sample areas.
Nevertheless, the LKJ and SAR priors are not as useful when there are out-of-sample regions, as the correlation matrix is assumed to have as many dimensions as in the training set.
In such cases, an autoregressive or a random walk prior on $u_d$ such as in \cite{gao_improving_2021} can be more appropriate, as it does not depend strictly on the dimensions of the correlation matrix.
In any case, it is implausible that the area-level effects are completely independent.
Therefore, the SAR specification is chosen as the final model.
Next section compares all models fitted until now using stacking weights.














\section{Comparison of Bayesian models with stacking weights}

Model selection is a uncertain procedure, especially when there are multiple plausible models that are consistent with the data (MCELREATH?).
While in section \ref{ch:area_corr} the SAR model was selected as the final model, it was also clear in Table \ref{tab:lkj_sar_base} that the model was indistinguishable from or only marginally better than the other two models.
Stacking, which was discussed in section \ref{ch:bayesian_evaluation}, provides a method to combine predictions from multiple heterogeneous models.
Weights are calculated jointly for all models based on their elpd$_{\text{loo}}$.
A weighted average of the predictions from the different models can be estimated.
The present paper does not investigate whether stacking provides better results than a single model.
However, the weights provide valuable insights into the predictive power of the different models.
First, stacking is relatively insensitive to similar models.
This can be seen in the first three models, which are identical with the exception of the tightness of the prior on the skewness.
The first three models share their weights in the sense that only the weight of the third model is non-zero.
The last three entries in Table \ref{tab:stacking} correspond to the three models used to introduce area-level correlation: a base model with no correlation, the LKJ model and the SAR model.
These three models offer a different picture: all of their weights are non-zero, which indicates that they are heterogenous, or else some of the weights would be equal to zero.
Additionally, their weights (0.25, 0.31, 0.24) are in a very similar range.
This indicates that they would contribute almost equally to a weighted average of predictions.
This confirms the results in Table \ref{tab:lkj_sar_base} that showed a very similar elpd$_{\text{loo}}$ for all three models.

% latex table generated in R 4.0.5 by xtable 1.8-4 package
% Sun Aug 22 09:31:33 2021
\begin{table}[ht]
    \caption{Stacking weights of models in the workflow.}
    \begin{center}
        \begin{tabular}{rrrrrrr}
          \hline
         Mid skew. & Low skew. & High skew. & Base & LKJ & SAR \\
          \hline
         0.00 & 0.00 & 0.19 & 0.25 & 0.31 & 0.24 \\
           \hline
        \end{tabular}
    \end{center}

    \begin{adjustwidth}{45pt}{45pt}
        \footnotesize{The first three models correspond to the three different priors on skewness from section \ref{ch:log_shift}. The \textit{Base} model corresponds to the model with no area-level correlation developed in section \ref{ch:coef_var_spec}. The \textit{LKJ} and \textit{SAR} models correspond to the models with area-level correlation from section \ref{ch:area_corr}. All models used the stratified random effect discussed in section \ref{ch:raneff}.}
    \end{adjustwidth}
    \label{tab:stacking}
\end{table}

In summary, the decision to use only one the SAR model is not necessarily the only possible one.
It is likely that a combined prediction with methods such a stacking will outperform the predictions from a single model, but this question is left for future research.
