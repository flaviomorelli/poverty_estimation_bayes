\begin{center}
    \LARGE
    \textbf{Abstract}
\end{center}
\vspace{2cm}

Reliable poverty estimates are the basis for public policy decision-making.
However, the estimation of poverty indicators is challenging not only due to the unimodal, leptokurtic and skewed nature of income data, but also because of small sample sizes in income surveys.
This paper applies the ideas of the Bayesian workflow proposed by \cite{gelman_bayesian_2020} to develop a hierarchical Bayesian (HB) model iteratively in the \textit{small area estimation} (SAE) context.
The main advantages of the Bayesian paradigm are modelling flexibility and a high-degree of model interpretability.
After comparing plausible modelling approaches, the Bayesian model is developed by defining adequate priors on regression coefficients and variance parameters, doing variable selection, redefining the specification of the random effect and including area-level correlation.
The resulting Bayesian model is then compared with the frequentist EBP approach under data-driven transformations \citep{rojas_perilla_data_2020}.

\vspace{3cm}

Die verlässliche Schätzung von Armutsindikatoren ist eine notwendige Grundlage für den politischen Entscheidungsfindungsprozess.
Die Schätzung von Armutsindikatoren ist jedoch nicht nur aufgrund der unimodalen, leptokurtischen und schiefen Verteilung von Einkommensdaten eine Herausforderung, sondern auch wegen der kleinen Stichprobengrößen bei Einkommenserhebungen.
Die vorliegende Arbeit wendet die Grundprinzipien des Bayesianischen Workflows aus \cite{gelman_bayesian_2020} an, um ein hierarchisches Bayesianisches (HB) Modell im Kontext von \textit{small area estimation} (SAE) iterativ zu entwickeln.
Die Vorteile des Bayesianischen Paradigmas sind zum einen die Flexibilität in der Modellierungsphase und zum anderen die Modellinterpretierbarkeit.
Unter Berücksichtigung verschiedener Modellierungsansätze werden das Bayesianische Modell entwickelt, passende A-Priori-Verteilungen für Regressionskoeffizienten und Varianzparameter definiert, eine Variablenselektion vorgenommen, die Spezifikation des \textit{random effect} geändert und die räumliche Korrelation einbezogen.
Das resultierende Bayesianische Modell wird im Abschluss mit dem frequentistischen EBP-Ansatz unter datengetriebenen Transformationen \citep{rojas_perilla_data_2020} verglichen.
