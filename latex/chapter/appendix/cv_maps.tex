\chapter{Appendix: Additional Maps}
\label{appendix:cv_maps}

This appendix presents additional maps with the coefficients of variation (CV).
For the HB model, the CV is defined as the standard deviation of the estimate over the estimate itself.
In the EBP, the RMSE of the estimate is used instead of the standard deviation.
The results are shown in Figure \ref{fig:cv_maps}.
All estimates from the HB model have an acceptable CV (<0.1).
A striking feature of this map is how different the patterns are in the HB and EBP.
Regions that have a higher CV in the HB map compared to other regions show the opposite behavior in the EBP map.
Also note that in general terms the CV of the EBP model has higher values, which is likely caused by the differences between RMSE and standard deviation.


\begin{figure}[h]
    \begin{subfigure}{0.49\linewidth}
        \centering
        \includegraphics[width=\textwidth]{./graphics/maps/hb_hcr_cv}
        \caption{HCR CV (HB)}
    \end{subfigure}
    \begin{subfigure}{0.49\linewidth}
        \centering
        \includegraphics[width=\textwidth]{./graphics/maps/ebp_hcr_cv}
        \caption{HCR CV (EBP)}
    \end{subfigure}

    \begin{subfigure}{0.49\linewidth}
        \centering
        \includegraphics[width=\textwidth]{./graphics/maps/hb_pgap_cv}
        \caption{PGAP CV (HB)}
    \end{subfigure}
    \begin{subfigure}{0.49\linewidth}
        \centering
        \includegraphics[width=\textwidth]{./graphics/maps/ebp_pgap_cv}
        \caption{PGAP CV (EBP)}
    \end{subfigure}
    \caption{Coefficient of variation for the HCR and PGAP indicators.}
    \label{fig:cv_maps}
\end{figure}

To better understand this pattern, it is useful to take a look at the out-of-sample map in Figure \ref{fig:guerrero_in}.
The out-of-sample pattern matches the regions with the highest CV in the EBP model, which reflects a higher uncertainty.
On the other hand, some regions that have the lowest CV in the HB model are out-of-sample.
This is another indication that redefining the random effect to avoid out-of-sample areas greatly reduces uncertainty in the estimates.


\begin{figure}
    \centering
    \includegraphics[width=\textwidth]{./graphics/maps/guerrero_in}
       \caption{Map of Guerrero. Light gray areas are out of sample, dark areas in sample.}
    \label{fig:guerrero_in}
\end{figure}