\chapter{Appendix: Jacobian Adjustment after Log-Shift Transformation}
\label{appendix:jacobian}

The transform of the dependent variable introduces a distortion that makes it necessary to adjust the Jacobian of the likelihood using Jacobi's transformation formula \cite[Theorem 12.6]{jacod_probability_2004}.
Let $X$ be a univariate random variable defined over $I \subseteq \mathbb{R}$ with density $f^X$. For a continuous differentiable function $\varphi: I \rightarrow \mathbb{R}$ and $\varphi' \ne 0$ for all $x \in I$, define $Y := \varphi(X)$.
The density of the random variable $Y$ can then be defined as
\begin{equation*}
    f^Y(y) = f^X(\varphi^{-1}(y))
    \left|\displaystyle \frac{d \varphi^{-1}(y)}{dy}\right|
    \mathbb{I}(y \in \varphi(I)), ~~ y \in \mathbb{R},
\end{equation*}
where $\mathbb{I}(\cdot)$ is the indicator function.
Assuming that $y$ represents the dependent variable in the original scale, then $\varphi^{-1}(y) = \log(y + \lambda)$ and the Jacobian adjustment is $\displaystyle \frac{d \varphi^{-1}(y)}{dy} = \frac{1}{y + \lambda}$, which is positive due to $y + \lambda > 0$. As the transformation parameter $\lambda$ is chosen to drastically reduce the skewness of $\log(y + \lambda)$, the density $f^X$ is chosen to be from a symmetric distribution – e.g., a Student's $t$-distribution. The likelihood of $y$ under the log-shift transformation is therefore
\begin{equation*}
    \displaystyle f^Y(y) = \text{Student}(\log(y + \lambda)| \mu , \sigma, \nu)
    \cdot \frac{1}{y + \lambda}
    \cdot\mathbb{I}(y \ge \lambda), ~~ y \in \mathbb{R}.
\end{equation*}