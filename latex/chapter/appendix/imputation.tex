\chapter{Appendix: Imputation of very extreme predictions}
\label{ch:imputation}
Fitting skewed data is challenging due to the long right tail of the distribution, which can lead to very high predictions.
This is especially the case when using the log-shift transformation combined with the Student's $t$-distribution, which has very heavy tails depending on the degrees of freedom.
In practice, these extreme predictions represent only a small proportion (under 0.5\%) of the samples from the posterior predictive distribution.
Still, to avoid problems with the posterior predictive checks a simple imputation procedure is implemented.
The first step is to check which predictions are 10\% or more above the dependent variable $y$ maximum and to mark them as missing.
Because the prediction has a very high value, the assumption is that it would still have a higher otherwise, just not in such an extreme range.
Therefore, those predictions are then drawn from $\mathcal U(q_{0.99}(y), \max y)$, i.e. a uniform distribution between the 99\%-quantile of $y$ and its maximum.
As an example, in the log-scale scenario the maximum of $y$ is around 30.000 while its 99\%-quantile is approximately 10.000.
While a uniform distribution might seem a crude choice for the imputation, the probability mass above the 99\% quantile is almost zero and it should therefore be an adequate approximation.
Nonetheless, it is certainly possible to use distributions that place less probability mass on higher values.
Finally, it is important to underline that this imputation procedure is a heuristic and that depending on the application it might make sense to use another procedure.