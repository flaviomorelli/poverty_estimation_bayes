% Introduction
\begin{frame}{Motivation}

\begin{itemize}
    \item Apply Bayesian workflow to poverty estimation \cite{gelman_bayesian_2020}
    \item Iterative model improvement
    \item Bayesian statistics allow for model flexibility
    \item Bayesian models provide tools to make assumptions transparent
\end{itemize}

\end{frame}


\begin{frame}{Terminology: Bayes' theorem}
    \begin{center}
        $p(\theta|y) = \displaystyle \frac{p(\theta, y)}{p(y)} = \displaystyle \frac{p(y|\theta) \cdot p(\theta)}{\int_\Theta p(y|\theta) \cdot p(\theta) d\theta}$\\
    \end{center}

\end{frame}

% BAYESIAN MODEL
\begin{frame}{Terminology: Bayes' theorem}
    \begin{center}
        $p(\theta|y) = \displaystyle \frac{p(\theta, y)}{p(y)} = \displaystyle \frac{p(y|\theta) \cdot p(\theta)}{\int_\Theta p(y|\theta) \cdot p(\theta) d\theta}$\\
    \end{center}

    \begin{itemize}


        \item
        $\theta \in \Theta$: a parameter or vector of parameters (e.g. the probability in a binomial distribution) in the parameter space $\Theta$

        \item
        $y$: data

        \item
        $p(\theta|y)$: posterior

        \item
        $p(y|\theta)$: likelihood function that captures how we are modeling our data stochastically (e.g. $y$ is a binomial variable)

        \item
        $p(\theta)$: prior knowledge about the parameters (e.g. a probability can only be between 0 and 1)



    \end{itemize}
\end{frame}



\begin{frame}{Terminology: Bayes' theorem}
    \begin{center}
        $p(\theta|y) = \displaystyle \frac{p(\theta, y)}{p(y)} = \displaystyle \frac{p(y|\theta) \cdot p(\theta)}{\int_\Theta p(y|\theta) \cdot p(\theta) d\theta}$\\
    \end{center}

    \begin{itemize}

        \item
        $p(y) = \int_\Theta p(y|\theta) \cdot p(\theta) d\theta$: the marginal likelihood or evidence

        \item
        $p(y)$ is a constant that normalizes the expression so that the posterior integrates to one.

        \item
        Problem: $p(y)$ is \textbf{intractable} even for very simple models, and this makes it very hard to calculate the posterior

        \item
        Bayesian computation methods get around this intractability by different means


    \end{itemize}

\end{frame}

\begin{frame}{SAE: Original HB model}
    \vspace{-1cm}
    Original HB model \cite{molina_small_2014}
    \begin{equation*}
        \label{eq:hb_rao}
        \begin{split}
            y_{di} |\boldsymbol \beta, u_d, \sigma_e & \sim \mathcal N(\boldsymbol{x'}_{di} \boldsymbol{\beta}+ u_d, \sigma_e), \\
            u_d | \sigma_u & \sim \mathcal N(0, \sigma_u), \\
            p(\boldsymbol \beta, \sigma_u, \sigma_e) & = p(\boldsymbol \beta) p(\sigma_u)p(\sigma_e) \propto p(\sigma_u)p(\sigma_e).
        \end{split}
    \end{equation*}
    Issues with the model:
    \begin{itemize}
        \item Sticks to the assumption of normal errors
        \item Flat priors on most parameters
        \item Parametrized to avoid MCMC
    \end{itemize}
\end{frame}

\begin{frame}{SAE: Modified HB model}
    Modified HB model \cite{morelli_hierarchical_2021}
    \begin{equation*}
        \begin{split}
            y_{di} |\boldsymbol \beta, u_d, \sigma_e & \sim
            \text{Student}(\boldsymbol{x'}_{di} \boldsymbol \beta + u_d,\ \sigma_e\ , \nu), \\
            u_d | \sigma_u & \sim \mathcal N(0, \sigma_u),\  \\
            \beta_k & \sim \mathcal N(\mu_k, \sigma_k),\\
            \sigma_u & \sim Ga(2, a), \\
            \sigma_e & \sim Ga(2, b), \\
            \nu & \sim Ga(2, 0.1). \\
        \end{split}
        \label{eq:mod_hb}
    \end{equation*}
    Differences with Molina model:
    \begin{itemize}
        \item Likelihood is a Student's $t$-distribution
        \item All priors are proper distributions
        \item Very hard to estimate without MCMC!
    \end{itemize}
\end{frame}

\begin{frame}{Poverty Indicators: FGT}
    \vspace{-2cm}
    \begin{gather*}
        F_d(\alpha, t) = \displaystyle \frac 1 {N_d} \sum_{i=1}^{N_d}\left( \frac{t - y_{di}}{t} \right)^\alpha I (y_{di} \le t),
        \hspace{1cm}\alpha = 0, 1, 2.
    \end{gather*}

    \begin{itemize}
        \item FGT-indicators \cite{foster_class_1984}
        \item $t$ is the poverty line set at 60\% of median income of the state
        \item $y_{di}$ is the income of the $i$-th person in municipality $d$
        \item $I(\cdot)$ is the indicator function
        \item $\alpha = 0$ is the head count ratio (HCR)
        \item $\alpha = 1$ is the poverty gap (PGAP)
        \item $\alpha = 2$ is the poverty severity
        \item Hard to estimate with regression only
    \end{itemize}
\end{frame}


\begin{frame}
    \vspace{-0.5cm}
    \centering
    \scalebox{.7}{

    \scriptsize
    \begin{algorithm*}[H]
        \caption{Estimate FGT-indicators with HB model}
        \KwIn{A model $p(\theta, y)$, some data $y$ and $\alpha \in \{0, 1, 2\} $}
        \KwOut{$\hat F_d^{HB}, \hat \sigma^{HB}_d,$ for $d = 1, ..., D$}
        \For{$s \in \{1,...,S\}$}{

            \For{$d \in \{1,...,D\}$}{
                $\tilde y_d^{(s)}|y = (\tilde y_{d1}^{(s)}, ..., \tilde y_{dN_d}^{(s)})'$\;
                Sample $\boldsymbol {\hat \beta^{(s)}}, \hat u_d^{(s)}, \hat \sigma_e^{(s)}, \hat \sigma_u^{(s)}, \hat \nu^{(s)}$ from $p(\beta, u_d, \sigma_e, \sigma_u, \nu|y)$\;
                \eIf{$d$ is in-sample}{
                    Sample $\tilde y^{(s)}_d|y$ from  $\text{Student}(\boldsymbol{x'}_{d} \boldsymbol {\hat \beta^{(s)}} + \hat u_d^{(s)}, \hat \sigma_e^{(s)}, \hat \nu^{(s)})$
                }{\If{$d$ is out-of-sample}{
                        Sample $\tilde u_d^{(s)}$ from $\mathcal N(0, \hat \sigma_u^{(s)})$\;
                        Sample $\tilde y^{(s)}_d|y$ from $\text{Student} (\boldsymbol{x'}_{d} \boldsymbol {\hat \beta^{(s)}} + \tilde u_d^{(s)}, \hat \sigma_e^{(s)}, \hat \nu^{(s)})$
                    }
                }
            }
            $\tilde y ^{(s)} = (y ^{(s)}_1,..., y ^{(s)}_D)'$\;
            $t^{(s)} = 0.6 \cdot median(\tilde y ^{(s)})$\;
            \For{$d \in \{1,...,D\}$}{
                $F^{(s)}_d(\alpha, t^{(s)}) = \displaystyle \frac 1 {N_d} \sum_{i=1}^{N_d}\left( \frac{t^{(s)} - \tilde y_{di}}{t^{(s)}} \right)^\alpha I (\tilde y_{di} \le t^{(s)})$
            }
        }

        \For{$d \in \{1,...,D\}$}{
            % Mean estimator
            $\hat F_d^{HB} = \displaystyle \frac 1 S \sum_{s = 1}^S F_d^{(s)}$\;
            % Std. deviation
            $\hat \sigma^{HB}_d = \displaystyle \sqrt{ \frac{1}{S-1}  \sum_{s = 1}^S \left( F_d^{(s)} - \hat F_d^{HB} \right)^2}$
        }
    \end{algorithm*}}
\end{frame}

\begin{frame}{Types of models}
    \begin{block}{Unit level models}
        \begin{itemize}

            \item Use the survey data directly

            \item Area level covariates will be needed to provide small area estimates

            \item Sometimes access to individual data is difficult because of problems
            of confidentiality

        \end{itemize}

        \begin{block}{Area level models}
            \begin{itemize}

                \item Based on direct estimates and area level covariates

                \item No confidentiality issues involved

                \item Given that these are based on ecological data, the coefficients
                of the covariates in the model must be interpreted with care

            \end{itemize}
        \end{block}
    \end{block}
    \setlength{\TPHorizModule}{1mm}
    \setlength{\TPVertModule}{1mm}
    \begin{textblock}{140}(3.28,89.7)
        \tiny{Reference: \cite{Jiang2006a}}
    \end{textblock}
\end{frame}

